% Preamble -------------------------------------------------
\documentclass{beamer}
\usepackage[utf8]{inputenc}
\usepackage[ngerman]{babel}
\usepackage{hyperref}
\usepackage{graphicx}
\usepackage{tikz}
  \usetikzlibrary{positioning, backgrounds, calc}
\usepackage{tikzscale}
\usepackage{adjustbox}
\usepackage{multirow}
\usepackage{booktabs}
% Slides setup ---------------------------------------------
\usetheme{Berlin}
\usecolortheme{seagull}
\usefonttheme{professionalfonts}

\title{Zusammenfassung vom 30.04.2018}
\author{Dag Tanneberg\thanks{%
  \href{mailto:dag.tanneberg@uni-potsdam.de}%
    {dag.tanneberg@uni-potsdam.de}
  }
}
\institute[Universität Potsdam]{
  {\glqq}Grundlagen der Vergleichenden Politikwissenschaft{\grqq}\\
  Universität Potsdam\\
  Lehrstuhl für Vergleichende Politikwissenschaft\\
  Sommersemester 2018
}
\date{07.05.2018}

\definecolor{grey538}{RGB}{240,240,240}
% document body --------------------------------------------
\begin{document}

\maketitle

\begin{frame}
  \frametitle{Leitfragen der Sitzung}
  \begin{enumerate}
    \item Was leistet der strukturell-individualistische Ansatz?
    \item Was zeichnet die Theorie der rationalen Wahl aus?
    \item Was ist eine rationale Wahl?
    \item Was habe ich davon?
  \end{enumerate}
\end{frame}

\begin{frame}
  \frametitle{Was leistet der strukturell-individualistische Ansatz?}
  \begin{itemize}
    \item \textbf{Zweck}: kollektive Explananda d. Individualverhalten erklären
    \item \textbf{Mittel}: Mehrebenenzshg. zw. Struktur \& Akteur
    \item \textbf{Analyseschritte}:
    \begin{enumerate}
      \item Situation: Welche Makromerkmale sind handlungsrelevant?
      \item Selektion: Wie wählen Individuen zw. Handlungsalternativen?
      \item Aggregation: Wie überlagern sich Handlungsentscheidungen?
    \end{enumerate}
    \item \textbf{Handlungstheorie}:
    \begin{itemize}
      \item benennt handlungsrelevante Situationsmkm.
      \item informiert Entsch. zw. Handlungsalternativen
    \end{itemize}
  \end{itemize}
\end{frame}

\begin{frame}
  \frametitle{Was leistet der strukturell-individualistische Ansatz?}
  \begin{figure}
  \centering
    \includegraphics{analyseschema.tikz}
  \end{figure}
\end{frame}

\begin{frame}
  \frametitle{Was zeichnet die Theorie der rationalen Wahl aus?}
  \begin{itemize}
    \item \textbf{zentral}: Bedürfnisbefriedigung unter Bdg. von Knappheit
    \item \textbf{Handeln}:
    \begin{itemize}
      \item Allokation knapper Mittel auf konkurrierende Ziele
      \item [$\rightarrow$] planvolle \& intentionale Wahlentschiedung unter Restriktionen
    \end{itemize}
    \item \textbf{Handlungsziel}: Allokation maximiert den Individualnutzen
    \item [$\rightarrow$] $\exists$ Rangfolge über Handlungskonsequenzen
  \end{itemize}
\end{frame}

\begin{frame}
\frametitle{Was ist eine rationale Wahl?}
\begin{itemize}
  \item \textit{Präferenzen über Handlungsfolgen} leiten Handlungsselektion
  \item erfordert eine kohärente Präferenzordnung
\end{itemize}
\begin{figure}[t]
\centering
\begin{tikzpicture}[grow = right, level distance = 1em, sibling distance = 4em, background rectangle/.style = {fill = grey538, draw = black}, show background rectangle]
% node placement
\node (0) [draw, shape = circle, fill = black, inner sep = 0.1em] {}
  child { node  (0-1) {} edge from parent [-latex]}
  child { node (0-2) {} edge from parent [-latex] }
;
% label placement
\node [left = 0em and 0em of 0] {
  \begin{tabular}{l}
  \textbf{Präferenz-}\\
  \textbf{ordnung}
  \end{tabular}
} ;
\node (l1) [right = 0em and 0em of 0-2] {
  \begin{tabular}{l}
    \textbf{Vollständigkeit}\\
    $\bullet$ erschöpfender Vergleich\\
    $\bullet ~ \forall ~ i, j \in I: i \ge j \lor i \le j \lor i = j$\\
  \end{tabular}
} ;
\node (l2) [right = 0em and 0em of 0-1] {
  \begin{tabular}{l}
    \textbf{Transitivität} \\
    $\bullet$ widerspruchsfreie Ordnung \\
    $\bullet ~ \forall ~ i, j, k \in I: i \ge j \land j \ge k \implies i \ge k$ \\
  \end{tabular}
} ;
\end{tikzpicture}
\end{figure}
\end{frame}

\begin{frame}
  \frametitle{Was habe ich davon?}
  \begin{enumerate}
    \item belastbare Analytik von Interaktionszusammenhängen
    \begin{itemize}
      \item Wer verfolgt welche Zwecke?
      \item Welche Mittel setzt der Akteur wahrscheinlich ein?
    \end{itemize}
    \item bildet widerspruchsfreie Theorien
    \begin{itemize}
      \item Zwingt zu transparenten Annahmen
      \item Fördert annahmentreue Argumentation
    \end{itemize}
    \item vielseitig einsetzbar \& empirisch erprobt
    \begin{itemize}
      \item Analysiert Ein- und Mehrpersonenzusammenhänge
      \item Bietet ein Portfoliot von Standardproblemen
    \end{itemize}
  \end{enumerate}
\end{frame}
\end{document}