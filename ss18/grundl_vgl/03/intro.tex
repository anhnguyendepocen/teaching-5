% Preamble -------------------------------------------------
\documentclass{beamer}
\usepackage[utf8]{inputenc}
\usepackage[ngerman]{babel}
\usepackage{hyperref}
\usepackage{graphicx}
\usepackage{tikz}
  \usetikzlibrary{positioning}
  \usetikzlibrary{calc}
  \usetikzlibrary{matrix}
\usepackage{adjustbox}
\usepackage{multirow}
\usepackage{booktabs}
% Slides setup ---------------------------------------------
\usetheme{Berlin}
\usecolortheme{seagull}
\usefonttheme{professionalfonts}

\title{Zusammenfassung vom 23.04.2018}
\author{Dag Tanneberg\thanks{%
  \href{mailto:dag.tanneberg@uni-potsdam.de}%
    {dag.tanneberg@uni-potsdam.de}
  }
}
\institute[Universität Potsdam]{
  {\glqq}Grundlagen der Vergleichenden Politikwissenschaft{\grqq}\\
  Universität Potsdam\\
  Lehrstuhl für Vergleichende Politikwissenschaft\\
  Sommersemester 2018
}
\date{30.04.2018}
% document body --------------------------------------------
\begin{document}

\maketitle

\begin{frame}
  \frametitle{Leitfragen der Sitzung}
  \begin{enumerate}
    \item Was sind verstehen und erklären?
    \item Was heißt vergleichen?
    \item Welche Varianten des Vergleichs gibt es?
  \end{enumerate}
\end{frame}

\begin{frame}
  \frametitle{Was sind verstehen und erklären?}
  \begin{itemize}
    \item \textbf{Verstehen}: idiographisch, d.h. Einzigartigkeit beschreiben
    \item \textbf{Erklären}: nomothetisch, d.h. Betrachtung generalisieren
    \item [$\rightarrow$] alternative Erkenntnisinteressen
    \item \textbf{Vergleichen}: Gemeinsamkeiten und Unterschiede begründen
    \item [$\rightarrow$] Einzigartigkeit eliminieren \& Generalisierung erreichen
  \end{itemize}
\end{frame}

\begin{frame}
  \frametitle{Was heißt vergleichen?}
  \begin{itemize}
    \item \textbf{Kontrolle ausüben}: Kann man eine Sache verallgemeinern?
    \item \textbf{Kriterien formulieren}: Kann man sinnvoll vergleichen?
    \item \textbf{Verallgemeinerung abwägen}: Wie wichtig sind Ideosynkrasien?
  \end{itemize}
\end{frame}

\begin{frame}
  \frametitle{Welche Varianten des Vergleichs gibt es?}
  \begin{itemize}
    \item \textbf{Mill's Methoden}: Logische Modelle des Vergleichs
    \item \textbf{Auswertung von Kovarianz}: eliminieren altern. Erkl.
    \item \textbf{Prototypische Forschungsdesigns}: MoA \& MoD
    \item \textbf{Scheitern an}: multipler Kausalität, Interaktionseffekten, probabilistischen Beziehungen
  \end{itemize}
  \begin{table}[b]
    \centering
    \begin{tabular}{l *{8}{c}}
      \toprule
      Method of & \multicolumn{4}{c}{Agreement} & \multicolumn{4}{c}{Difference}\\
      \cmidrule(lr){2-5} \cmidrule(lr){6-9}
      & $x_1$ & $x_2$ & $x_3$ & $y$ & $x_1$ & $x_2$ & $x_3$ & $y$ \\
      \midrule
      & 1 & 0 & 1 & 1 & 1 & 1 & 0 & 1\\
      & 1 & 1 & 0 & 1 & 0 & 1 & 0 & 0\\
      \bottomrule
    \end{tabular}
  \end{table}
\end{frame}
\end{document}