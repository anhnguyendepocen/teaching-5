% Preamble -------------------------------------------------
\documentclass{beamer}
\usepackage[utf8]{inputenc}
\usepackage[ngerman]{babel}
\usepackage{hyperref}
\usepackage{graphicx}
\usepackage{tikz}
  \usetikzlibrary{positioning}
  \usetikzlibrary{calc}
  \usetikzlibrary{matrix}
\usepackage{adjustbox}
\usepackage{multirow}
\usepackage{booktabs}
% Slides setup ---------------------------------------------
\usetheme{Berlin}
\usecolortheme{seagull}
\usefonttheme{professionalfonts}

\title{Zusammenfassung vom 04.06.2018}
\author{Dag Tanneberg\thanks{%
  \href{mailto:dag.tanneberg@uni-potsdam.de}%
    {dag.tanneberg@uni-potsdam.de}
  }
}
\institute[Universität Potsdam]{
  {\glqq}Grundlagen der Vergleichenden Politikwissenschaft{\grqq}\\
  Universität Potsdam\\
  Lehrstuhl für Vergleichende Politikwissenschaft\\
  Sommersemester 2018
}
\date{11.06.2018}
% document body --------------------------------------------
\begin{document}

\maketitle

\begin{frame}
  \frametitle{Leitfragen der Sitzung}
  \begin{enumerate}
    \item Was sind räumliche Modelle der Politik?
    \item Aus welchen Bausteinen bestehen diese Modelle?
    \item Was kann ich mit ihnen anfangen?
  \end{enumerate}
\end{frame}

\begin{frame}
  \frametitle{Was sind räumliche Modelle der Politik?}
  \begin{itemize}
    \item Abbildung der Politik auf euklidische Räume
    \begin{itemize}
      \item [$\rightarrow$] vereinfacht: n-dimensionale Koordinatensysteme
    \end{itemize}
    \item Analyse resultierender Akteurskonstellationen
      \begin{itemize}
      \item [$\rightarrow$] Theorie der rationalen Wahl
    \end{itemize}
    \item Klassische Beiträge
    \begin{itemize}
      \item Hotellings räumlicher Wettbewerb (1929)
      \item Blacks Median-Wähler-Theorem (1948)
      \item Downs Ökonomische Theorie der Demokratie (1957)
    \end{itemize}
  \end{itemize}
\end{frame}

\begin{frame}
  \frametitle{Aus welchen Bausteinen bestehen diese Modelle?}
  \begin{enumerate}
    \item Basisannahmen der Theorie der rationalen Wahl
    \begin{itemize}
      \item vollständige \& transitive Präferenzordnung
    \end{itemize}
    \item Eingipfeligkeit (Single-peakedness) von Präferenzordnungen
    \begin{itemize}
      \item Idealpunkt: $y_i \succ o ~~ \forall ~~ o \in O\setminus\{y_i\}$
    \end{itemize}
    \item Nutzenfunktion $u$ über Politikbündel im n-dimens. Raum
    \begin{enumerate}
      \item $u$ hat ein Maximum am Idealpunkt des Akteurs
      \item $u$ der Wert von $u$ nimmt jenseits des Idealpunkts ab
      \item häufig: $u$ ist symmetrisch
    \end{enumerate}
    \item Salienzvektor
    \begin{itemize}
      \item Aussage über die Wichtigkeit einzelner Politikfelder
    \end{itemize}
    \item Lösungskonzept
    \begin{itemize}
      \item z.B. Gewinnmenge
      \item [$\rightarrow$] alle Punkte, die eine Koalition einem Referenzpunkt vorzieht
    \end{itemize}
  \end{enumerate}
\end{frame}

\begin{frame}
  \frametitle{Was kann ich mit ihnen anfangen?}
  \begin{itemize}
    \item sparsame, idealisierte Modellierung politischer Prozesse
    \item Vorhersagen über Wahlkampfverhalten, Koalitionsbildung, \dots
    \item Abweichung vom Modell $\implies$ Anstoß weiterer Theoriebildung
  \end{itemize}
\end{frame}
\end{document}