% Preamble -------------------------------------------------
\PassOptionsToPackage{table}{xcolor}
\documentclass{beamer}
\usepackage[ngerman]{babel}
\usepackage[utf8]{inputenc}
\usepackage[ngerman]{babel}
\usepackage{adjustbox}
\usepackage{tikz}
  \usetikzlibrary{positioning, calc, decorations.pathreplacing, backgrounds, fit}
\usepackage{multirow}
\usepackage{graphicx}
\usepackage[most]{tcolorbox}
\usepackage{booktabs}

% Slides setup ---------------------------------------------
\usetheme{Berlin}
\usecolortheme{seagull}
\usefonttheme{professionalfonts}

\title{Zusammenfassung vom 2. Juli 2018}
\author{Dag Tanneberg\thanks{%
  \href{mailto:dag.tanneberg@uni-potsdam.de}%
    {dag.tanneberg@uni-potsdam.de}
  }
}
\institute[Universität Potsdam]{
  {\glqq}Grundlagen der Vergleichenden Politikwissenschaft{\grqq}\\
  Universität Potsdam\\
  Lehrstuhl für Vergleichende Politikwissenschaft\\
  Sommersemester 2018
}
\date{9. Juli 2018}

% slides ---------------------------------------------------
\begin{document}
\maketitle

\begin{frame}
\frametitle{Leitfragen der Sitzung}
\begin{enumerate}
  \item Warum schreibt Lijphardt ``Patterns of democracy''?
  \item Welche grundlegende Frage motiviert seine Arbeit?
  \item Welche Demokratietypen unterscheidet Lijphardt?
  \item Kritik an Lijphardt
\end{enumerate}
\end{frame}

\begin{frame}
\frametitle{Warum schreibt Lijphardt ``Patterns of democracy''}
\begin{itemize}
  \item Vielfalt repräsentativer Demokratie
  \item Wiederkehrende Muster institutioneller Konfigurationen
  \item [$\rightarrow$] Muster beschreiben (Typologie) \&
    politische Bdtg. darlegen
\end{itemize}
\end{frame}

\begin{frame}
\frametitle{Welche grundlegende Frage motiviert seine Arbeit?}
  ``[W]ho will do the governing and to whose
  interests should the government be responsive when the
  people are in disagreement and have divergent
  preferences?''
  (Lijphardt 1999, 1)

\begin{enumerate}
  \item \textbf{Die Mehrheit!} \newline
    Mehrheitsdem. $\rightarrow$ Machtkonzentration in knappen
    Mehrheiten
  \item \textbf{So viele wie möglich!} \newline
    Konsensdem. $\rightarrow$ Teilung, Streuung und Brechung
    polit. Macht
\end{enumerate}

  ``[\dots] [T]he majoritarian model of democracy is exclusive,
  competitive, and adversarial, whereas the consensus model
  is characterized by inclusiveness, bargaining, and
  compromise; [\dots].'' (Lijphardt 1999, 2)
\end{frame}

\begin{frame}
\frametitle{Welche Demokratietypen unterscheidet Lijphardt}
\begin{tabular}{lll}
~   & \textbf{Konsensdem.} & \textbf{Mehrheitsdem.} \\ \hline
~   & Exekutive-Parteien \\ \hline
Regierungstyp  & Koalitionsreg. & Einparteireg. \\
Regierungsdominanz  & Nein & Ja \\
Parteiensystem  & Vielparteiens. & Zweiparteiens. \\
Wahlsystem & Verhältnisw. & Mehrheitsw. \\
Interessengruppensyst.  & Korporatismus & Pluralismus \\ \hline
    & \multicolumn{2}{l}{Föderalismus-Unitarismus} \\ \hline
Föderalismus & Ja & Nein \\
Bikameralismus  & Ja (stark) & Nein \\
Rigide Verf.  & Ja & Nein \\
Verfassungsger.  & Ja & Nein \\
Unabh. Zentralbank & Ja & Nein \\
\end{tabular}
\end{frame}

\begin{frame}
\frametitle{Kritik an Lijphardt}
\begin{itemize}
  \item nicht Mehrheit, sondern Machtkonzentration zentral
  \item Merkmale z.\,T. fragwürdig
  \begin{itemize}
    \item Wen repräsentiert die Zentralbank?
  \end{itemize}
  \item Merkmale kausal miteinander verknüpft\newline
    (Ganghof, VL Demokratietypen, Slide 22f.)
\end{itemize}
\begin{figure}
\centering
\begin{tikzpicture}[
  level distance = 9em,
  sibling distance = 2em,
  edge from parent/.style={draw,-latex}
]
\node{Wahlsysteme} [grow = right]
  child{
    node {Parteiensysteme}
    child{
      node {Regierungsdom.}
    }
    child{
      node {Regierungstyp}
    }
  }
;
\end{tikzpicture}
\vfill
\begin{figure}
\centering
\begin{tikzpicture}[
  level distance = 12em,
   sibling distance = 2em,
   edge from parent/.style={draw,-latex}
]
\node{Föderalismus} [grow = right]
  child{node {Verfassungsgerichstsbarkeit}}
  child{node {Bikameralismus}}
  child{node {Rigide Verfassung}}
;
\end{tikzpicture}
\end{figure}
\end{figure}
\end{frame}
\end{document}