% Preamble -------------------------------------------------
\PassOptionsToPackage{table}{xcolor}
\documentclass{beamer}
\usepackage[utf8]{inputenc}
\usepackage[ngerman]{babel}
\usepackage{adjustbox}
\usepackage{tikz}
  \usetikzlibrary{positioning, calc, decorations.pathreplacing, backgrounds, fit}
\usepackage{multirow}
\usepackage{graphicx}
\usepackage[most]{tcolorbox}
\usepackage{booktabs}

% Slides setup ---------------------------------------------
\usetheme{Berlin}
\usecolortheme{seagull}
\usefonttheme{professionalfonts}

\title{Rückmeldung zu den Diskussionspapieren}
\author{Dag Tanneberg\thanks{%
  \href{mailto:dag.tanneberg@uni-potsdam.de}%
    {dag.tanneberg@uni-potsdam.de}
  }
}
\institute[Universität Potsdam]{
  {\glqq}Grundlagen der Vergleichenden Politikwissenschaft{\grqq}\\
  Universität Potsdam\\
  Lehrstuhl für Vergleichende Politikwissenschaft\\
  Sommersemester 2018
}
\date{\today}

% slides ---------------------------------------------------
\begin{document}
\maketitle

\begin{frame}
\frametitle{Leitfragen der Sitzung}
\begin{enumerate}
  \item Wozu dient ein Diskussionspapier?
  \item Was lieft gut?
  \item Was lief schlecht?
\end{enumerate}
\end{frame}

\begin{frame}
\frametitle{Wozu dient ein Diskussionspapier?}
Das Diskussionspapier\dots
\begin{itemize}
  \item vertieft die Auseinandersetzung mit dem Seminarthema;
  \item problematisiert zentrale Debatten;
  \item trainiert die Analyse mit konkurrierenden Positionen;
  \item zwingt zur Formulierung eigener Gedanken.
\end{itemize}
\end{frame}

\begin{frame}
\frametitle{Was lief gut?}
Fast alle Papiere\dots
\begin{itemize}
  \item äußern sich einleitend zur Relevanz des Themas;
  \item trennen Darstellung und Erörterung;
  \item bilanzieren selbstständig die Kontroverse;
  \item diskutieren mit Hilfe eigener Beispiele.
\end{itemize}
\end{frame}

\begin{frame}
\frametitle{Was lief schlecht?}
\begin{itemize}
  \item Die Darstellung bleibt teilweise oberflächlich.
  \begin{itemize}
    \footnotesize
    \item Welche Koalitionsformate sagen \textit{vote}, \textit{office} oder \textit{policy seeking} vorher?
  \end{itemize}
  \item Schlussfolgerungen resultieren nicht aus den Annahmen.
  \begin{quote}
    \footnotesize
    Zudem zeichnet sich eine Koalition von inhaltlich nahen
    Parteien durch höhere Legitimität aus, was wiederum
    bedeutet, dass im Zusammenspiel mit einem gelungenen
    Koalitionsvertrag eine Wiederwahl der Koalitionsparteien
    deutlich höher ist, da sie erfolgreicher ihre Ziele
    umsetzen.
  \end{quote}
  \item Die Argumentation ignoriert Zielkonflikte.
  \begin{quote}
  \footnotesize
    Das Verhalten von Parteien in Koalitionsverhandlungen kann
    also [\dots nur; D.\,T.]
    durch eine Kombination [\dots] dieser Modelle erklärt
    werden, da Parteien zumindest aus instrumentellen
    Beweggründen sowohl Office- als auch Policy-Ziele
    verfolgen werden und die Grundlage für beide jeweils die
    Stimmenmaximierung ist.
  \end{quote}
\end{itemize}
\end{frame}
\end{document}