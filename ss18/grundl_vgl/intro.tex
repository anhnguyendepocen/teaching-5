% Preamble -------------------------------------------------
\PassOptionsToPackage{table}{xcolor}
\documentclass{beamer}
\usepackage[utf8]{inputenc}
\usepackage[ngerman]{babel}
\usepackage{adjustbox}
\usepackage{tikz}
  \usetikzlibrary{positioning, calc, decorations.pathreplacing, backgrounds, fit}
\usepackage{multirow}
\usepackage{graphicx}
\usepackage[most]{tcolorbox}

% Slides setup ---------------------------------------------
\usetheme{Berlin}
\usecolortheme{seagull}
\usefonttheme{professionalfonts}

\title{Zusammenfassung vom 18. Juni 2018}
\author{Dag Tanneberg\thanks{%
  \href{mailto:dag.tanneberg@uni-potsdam.de}%
    {dag.tanneberg@uni-potsdam.de}
  }
}
\institute[Universität Potsdam]{
  {\glqq}Grundlagen der Vergleichenden Politikwissenschaft{\grqq}\\
  Universität Potsdam\\
  Lehrstuhl für Vergleichende Politikwissenschaft\\
  Sommersemester 2018
}
\date{25. Juni 2018}

% slides ---------------------------------------------------
\begin{document}
\maketitle

\begin{frame}
\frametitle{Ausgangspunkt}

\textbf{Wahlen in Demokratien}
\begin{itemize}
  \item stellen allgemeinste Partizipationsform dar
  \item legitimieren Herrschaft in repr. Demokratien
\end{itemize}

\textbf{Leitfragen der Sitzung}
\begin{enumerate}
  \item Definition, Aufbau und Varianten von Wahlsystemen
  \item Wie arbeitet ein Wahl System?
  \item Was hat es mit Duvergers Gesetz \& Hypothese auf sich?
\end{enumerate}
\end{frame}

\begin{frame}
\frametitle{Politische Bdtg. von Wahlsystemen}
\textbf{Das Wahlsystem}
\begin{itemize}
  \item \textbf{Def.}: Regeln zur Übertragung von Päferemzen in Sitze
  \item \textbf{Zweck}: Rekrutierung polit. Ämter \& Repräsentativvers.
\end{itemize}
\textbf{Zielkonflikt}
\begin{itemize}
  \item Proportionales Ergebnis vs. Verantwortung f. Entscheidungen
  \item Wie viele polit. Parteien sollen berücksichtigt werden?
  \begin{enumerate}
    \item So viele wie nötig. $\rightarrow$ Geringe Disproportionalität
    \item So wenig wie möglich. $\rightarrow$ Eindeutige polit. Verantwortung
  \end{enumerate}
\end{itemize}
\end{frame}

\begin{frame}
\frametitle{Elementare Bausteine eines Wahlsystems}
\begin{enumerate}
  \item \textbf{Wahlkreisgröße}: Wie viele Mandate sind in einem Wahlkreis zu vergeben?
  \item \textbf{Verrechnungsregel}: Nach welchem Verfahren werden Stimmen in Sitze übertragen?
  \item \textbf{Sperrklauseln}: Ab welchem Stimmerergebnis werden Parteien bei der Sitzverteilung berücksichtigt?
  \item \textbf{Wahlsegmente}: Auf wie vielen Ebenen wird gleichzeitig gewählt?
  \item \textbf{Parteilisten}: Können \textit{Wähler} das Kandidatenangebot beeinflussen?
\end{enumerate}
\end{frame}

\begin{frame}
\frametitle{Wichtige Varianten von Wahlsystemen}
\begin{figure}[t]
\centering
\begin{tikzpicture}[edge from parent/.append style = {-latex}, sibling distance = 9em]
  \tikzstyle{node} = [draw, fill = black, circle, inner sep = .125em]
  \node (0) [node] {}
    child { node (0-1) {Mehrheitsw.}
      child [sibling distance = 5em] { node (0-1-1) {Absolut}
        % child { node (0-1-1-1) {\begin{tabular}{l}Two round\\double ballot\end{tabular}} }
        % child { node (0-1-1-2) {\begin{tabular}{l}Alterna-\\tive vote\end{tabular}} }
      }
      child [sibling distance = 5em] { node (0-1-2) {Relativ}
        % child { node (0-1-2-1) {\begin{tabular}{l}First past\\the post\end{tabular}} }
      }
    }
    child { node (0-2) {Gemischte WS.}
      child [sibling distance = 5em] {node (0-2-1) {\begin{tabular}{c}Segmen-\\tiert\end{tabular}} }
      child [sibling distance = 5em] {node (0-2-2) {\begin{tabular}{c}Kompen-\\satorisch\end{tabular}} }
    }
    child { node (0-3) {Verhältnisw.}
      child [sibling distance = 5em] {node (0-3-1) {\begin{tabular}{c}Listen-\\wahl\end{tabular}} }
      child [sibling distance = 5em] {node (0-3-2) {\begin{tabular}{c}Übertragbare\\Einzelst.\end{tabular}} }
    } ;
\begin{scope}[on background layer]
  \node (r1) [fill = black!15, fit = (0) (0-1-1) (0-3-2)] {};
\end{scope}
% \draw [-latex] (0) -- (0-1) node [midway, sloped, anchor = south] {Wahlkreisgr. $= 1$} ;
% \draw [-latex] (0) -- (0-3) node [midway, sloped, anchor = south] {Wahlkreisgr. $> 1$} ;
\end{tikzpicture}
\end{figure}
\end{frame}

\begin{frame}
  \frametitle{Wie arbeitet ein Wahl System?}
  \begin{enumerate}
    \item \textbf{mechanischer Effekt}: technische Regeln d. Mandatsvergabe
    \begin{itemize}
      \item Wie viele Parteien erringen ein Mandat?
      \item Wahlkreisgröße zentral
      \item arbeitet deterministisch
    \end{itemize}
    \item \textbf{psycholog. Effekt}: Antizipation von 1 durch Wähler \& Eliten
    \begin{itemize}
      \item Wie viele Parteien bewerben sich auf ein Mandat?
      \item bspw. Stimmensplitting bei Wahlen zum Bundestag
      \item arbeitet probabilistisch
    \end{itemize}
  \item [$\rightarrow$] psycholog. Effekt setzt den mechanischen E. voraus
  \item [$\rightarrow$] Effekte können in der Realität kaum getrennt werden
  \end{enumerate}
\end{frame}

\begin{frame}
  \frametitle{Was hat es mit Duvergers Gesetz \& Hypothese auf sich?}
  \begin{enumerate}
    \item Duvergers Gesetz
    \begin{itemize}
      \item Absolute Mehrheit in Einerwahlkreisen $\rightarrow$ Zweiparteiensystem
      \item Wähler stimmen für aussichtsreiche Parteien
      \item Ohne Erfolgsaussicht treten Eliten nicht in den Wettbewerb ein
    \end{itemize}
    \item Duvergers Hypothese
    \begin{itemize}
      \item VW \& MW mit 2. Runder $\rightarrow$ Vielparteiensystem
      \item größere Anzahl aussichtsreicher Parteien
      \item entspannt Selektionsdruck für Wähler \& Elite
    \end{itemize}
  \end{enumerate}
\end{frame}
\end{document}