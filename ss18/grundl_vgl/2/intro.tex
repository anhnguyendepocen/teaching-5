% Preamble -------------------------------------------------
\documentclass{beamer}
\usepackage[utf8]{inputenc}
\usepackage[ngerman]{babel}
\usepackage{hyperref}
\usepackage{graphicx}
\usepackage{tikz}
  \usetikzlibrary{positioning}
  \usetikzlibrary{calc}
  \usetikzlibrary{matrix}
\usepackage{adjustbox}

% Slides setup ---------------------------------------------
\usetheme{Berlin}
\usecolortheme{seagull}
\usefonttheme{professionalfonts}

\title{Zusammenfassung vom 16.04.2018}
\author{Dag Tanneberg\thanks{%
  \href{mailto:dag.tanneberg@uni-potsdam.de}%
    {dag.tanneberg@uni-potsdam.de}
  }
}
\institute[Universität Potsdam]{
  {\glqq}Grundlagen der Vergleichenden Politikwissenschaft{\grqq}\\
  Universität Potsdam\\
  Lehrstuhl für Vergleichende Politikwissenschaft\\
  Sommersemester 2018
}
\date{23.04.2018}
% document body --------------------------------------------
\begin{document}

\maketitle

\begin{frame}
  \frametitle{Fragen der Sitzung}
  \begin{enumerate}
    \item Was ist Wissenschaft?
    \item Wie läuft Wissenschaft ab?
    \item Warum betont die Wissenschaft Falsifikation?
  \end{enumerate}
\end{frame}

\begin{frame}
  \frametitle{Was ist Wissenschaft?}
  \begin{itemize}
    \item kritische Methode des vorläufigen Erkenntnisgewinns
    \item Wesentliche Merkmale:
    \begin{enumerate}
      \item Suche nach neuen Implikationen
      \item Versuch der Falsifikation
      \item Test konkurrierender Hypothesen
    \end{enumerate}
  \end{itemize}
\end{frame}

\begin{frame}
  \frametitle{Wie läuft Wissenschaft ab?}
\begin{adjustbox}{max totalsize={.9\textwidth}{.9\textheight},center}\begin{tikzpicture}
\def \n {5}
\def \radius {6em}
\def \margin {8} % margin in angles, depends on the radius
\foreach \s in {1,...,\n}
{
  \node [draw, circle, fill = white!75!blue](\s) at ({-360/\n * (\s - 1)}:\radius) {$\s$};
  \draw[<-, >=latex] ({360/\n * (\s - 1)+\margin}:\radius) 
    arc ({360/\n * (\s - 1)+\margin}:{360/\n * (\s)-\margin}:\radius);
}
\node [right = of 2] {
  \begin{tabular}{l}
    \underline{Theorie}\\
    $\bullet$ Widerspruchsfreiheit\\
    $\bullet$ mind. 1 emp. Implikation\\
  \end{tabular}
} ;
\node [left = of 3] {
  \begin{tabular}{l}
    \underline{Hypothese(n)}\\
    $\bullet$ falsifizierbar\\
    $\bullet$ möglichst zahlreich\\
  \end{tabular}
} ;
\node [left = of 4] {
  \begin{tabular}{l}
    \underline{Hypothesentest}\\
    $\bullet$ möglichst vielfältig\\
    $\bullet$ möglichst kritisch\\
  \end{tabular}
} ;
\node [right = of 5] {
  \begin{tabular}{l}
    \underline{Evaluation}\\
    $\bullet$ Theorie zurückweisen?\\
    $\bullet$ Neue Fragestellung?\\
  \end{tabular}
} ;
\node [right = of 1] {
  \begin{tabular}{l}
    \underline{Fragestellung}\\
    $\bullet$ Motivation\\
    $\bullet$ oft normativ inspiriert
  \end{tabular}
} ;
\end{tikzpicture}
\end{adjustbox}
\end{frame}

\begin{frame}
  \frametitle{Warum betont die Wissenschaft Falsifikation?}
  \textbf{Was ist Falsifikation?}\newline
  {\glqq}Vorgang oder Ergebnis der wissenschaftlichen Widerlegung
  von Aussagen, Hypothese oder Theorien{\grqq} (Nohlen \& Schultze 2004: Bd. 1, S. 228)
  \vfill
  \textbf{Welche dieser Aussagen sind falsifizierbar?}
  \begin{itemize}
    \item Rauchen erhöht die Wahrscheinlichkeit einer Krebserkrankung.
    \item Macht bedeutet jede Chance, innerhalb einer
      sozialen Beziehung den eignen Willen auch gegen
      Widerstreben durchzusetzen, gleichviel worauf diese
      Chance beruht. Macht ist sozial amorph.
    \item Wenn die Antiterrorgesetzgebung erfolgreich
      umgesetzt wird, dann werden wir nicht attackiert.
  \end{itemize}
\end{frame}

\begin{frame}
  \frametitle{Warum betont die Wissenschaft Falsifikation?}
  \begin{itemize}
    \item Hypothesen werden aus Theorien abgeleitet
    \item Wissenschaft prüft Implikationsaussagen
  \end{itemize}
  \begin{tabular}{*{4}{l}}
    ~ & Allgemeine Form & Beispiel \\ \hline
    Praemissa maior & Wenn $P$, dann $Q$. & Wenn es Winter ist, \\
    & &  dann regnet es in England.\\
    Praemissa minor & $P$. & Es ist Winter. \\ \hline
    Conclusio & Also $Q$. & Also regnet es in England. \\
  \end{tabular}
\end{frame}


% - i.d.R. Induktionsschluss von Daten auf allgemeine Theorien
% - Induktionsschluss nicht abschließend beweisbar
% - Falsifikation logisch eindeutiger

\begin{frame}
  \frametitle{Warum betont die Wissenschaft Falsifikation?}
  Das Antezedens ist in der Regel \textcolor{gray}{unsicher}.
  Wir müssen es aus beobachteten Konsequenz erschließen.
  \begin{table}
    \centering
    \begin{tabular}{*{3}{c}}
      Wenn es Winter ist, & dann regnet es in England. \\
      Antezedens & Konsequenz & Rückschluss \\ \hline
      \textcolor{gray}{W} & W & ungültig\\
      \textcolor{gray}{F} & W & \\ \hline 
      \textcolor{gray}{W} & F & gültig\\
      \textcolor{gray}{F} & F & \\
    \end{tabular}
  \end{table}
\end{frame}

\end{document}