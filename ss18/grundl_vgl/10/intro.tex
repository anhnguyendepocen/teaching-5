% Preamble -------------------------------------------------
\documentclass{beamer}
\usepackage[utf8]{inputenc}
\usepackage[ngerman]{babel}
\usepackage{hyperref}
\usepackage{graphicx}
\usepackage{tikz}
  \usetikzlibrary{positioning}
  \usetikzlibrary{calc}
  \usetikzlibrary{matrix}
\usepackage{adjustbox}
\usepackage{multirow}
\usepackage{booktabs}
% Slides setup ---------------------------------------------
\usetheme{Berlin}
\usecolortheme{seagull}
\usefonttheme{professionalfonts}

\title{Zusammenfassung vom 11.06.2018}
\author{Dag Tanneberg\thanks{%
  \href{mailto:dag.tanneberg@uni-potsdam.de}%
    {dag.tanneberg@uni-potsdam.de}
  }
}
\institute[Universität Potsdam]{
  {\glqq}Grundlagen der Vergleichenden Politikwissenschaft{\grqq}\\
  Universität Potsdam\\
  Lehrstuhl für Vergleichende Politikwissenschaft\\
  Sommersemester 2018
}
\date{18.06.2018}
% document body --------------------------------------------
\begin{document}

\maketitle

\begin{frame}
  \frametitle{Leitfragen der Sitzung}
  \begin{enumerate}
    \item Was leistet eine politische Partei?
    \item Wie beschreibt man Parteiensysteme?
    \item Wie entstehen Parteiensysteme?
  \end{enumerate}
\end{frame}

\begin{frame}
  \frametitle{Was leistet eine politische Partei?}
  \begin{itemize}
    \item \textbf{Definition}: ``a \textit{group of officials or would-be officials} who are linked with a sizeable \textit{group of citizens} into an \textit{organization}; a chief object of this organization is to ensure that its officials \textit{attain power or are maintained in power}'' (586; m.Hv.)
    \item \textbf{Analyseebenen}: Amt; Organisation; Elektorat
    \item \textbf{Funktionen}:
    \begin{enumerate}
      \item Politik vorstrukturieren
      \item polit. Personal rekrutieren \& sozialisieren
      \item Elektorat mobilisieren
      \item Eliten \& Wähler verbinden
    \end{enumerate}
  \end{itemize}
\end{frame}

\begin{frame}
  \frametitle{Wie beschreibt man Parteiensysteme?}
  \begin{itemize}
    \item \textbf{Definition}: ``Beziehungsgefüge der in einem polit. Gemeinwesen agierenden Parteien'' (Nohlen \& Schulze 2004: 635)
    \item \textbf{Merkmale}: Anzahl der Parteien, Größenverhältnisse, ideologische Distanzen, Interaktionsmuster, etc.
  \end{itemize}
  \begin{columns}
    \begin{column}{.4\linewidth}
    \begin{itemize}
      \item [] \textbf{Typologie}
      \item Keinparteisystem
      \item Einparteisystem
      \item Mehrps. mit dom. P.
      \item Zweiparteiensystem
      \item Mehrparteiensystem
    \end{itemize}
    \end{column}
    \begin{column}{.6\linewidth}
      \begin{itemize}
        \item [] \textbf{Effektive Parteienzahl}
        \item Kompromiss zw. Anzahl \& Größe
        \item Zählung \textit{relevanter} Parteien
        \item $N_{\text{Eff}} = \sum_{i=1}^N v_i^{-2}$
        \item $N_{\text{Eff}} = N \leftrightarrow v_i = v_j \forall i,j \in N$
        \item Varianten: elektoral; parlament.
      \end{itemize}
    \end{column}
  \end{columns}
\end{frame}

\begin{frame}
  \frametitle{Wie entstehen Parteiensysteme?}
  \begin{itemize}
    \item Parteien ruhen auf sozialen Cleavages auf
    \item [$\rightarrow$] tiefe Konfliktlinien in einer Gesellschaft
    \item Klassische Konfliktlinien nach Stein Rokkan:
    \begin{enumerate}
      \item Zentrum vs. Peripherie
      \item Staat vs. Kirche
      \item Kapital vs. Arbeit
      \item Agrarwirtschaft vs. Industrie
    \end{enumerate}
    \item Entstehung von Parteien:
    \begin{enumerate}
      \item primordial: Parteien repräsentieren eine präexistente Konflikte
      \item instrumental: polit. Unternehmer aktivieren latente Konflikte
    \end{enumerate}
  \end{itemize}
\end{frame}
\end{document}