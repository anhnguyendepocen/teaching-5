% Preamble -------------------------------------------------
\documentclass{beamer}
\usepackage[utf8]{inputenc}
\usepackage[ngerman]{babel}
\usepackage{hyperref}
\usepackage{graphicx}
\usepackage{tikz}
  \usetikzlibrary{positioning}
  \usetikzlibrary{calc}
  \usetikzlibrary{matrix}
\usepackage{adjustbox}

% Slides setup ---------------------------------------------
\usetheme{Berlin}
\usecolortheme{seagull}
\usefonttheme{professionalfonts}

\title{Zusammenfassung vom 23.04.2018}
\author{Dag Tanneberg\thanks{%
  \href{mailto:dag.tanneberg@uni-potsdam.de}%
    {dag.tanneberg@uni-potsdam.de}
  }
}
\institute[Universität Potsdam]{
  {\glqq}Wie erkl\"art man autorit\"are Herrschaft?{\grqq}\\
  Universität Potsdam\\
  Lehrstuhl für Vergleichende Politikwissenschaft\\
  Sommersemester 2018
}
\date{30.04.2018}
% document body --------------------------------------------
\begin{document}

\maketitle

\begin{frame}
  \frametitle{Leitfragen der Sitzung}
  \begin{enumerate}
    \item Gibt es einen eindeutigen Unterschied zwischen
      Demokratie und Autokratie?
    \item Sollten subjektive oder objektive Merkmale
      politischer Regime den Ausschlag für die Klassifikation
      geben?
    \item Welche Rolle spielen normative Erwägungen bei der
      Grenzziehung zwischen Demokratie und Autokratie?
  \end{enumerate}
\end{frame}

\begin{frame}
  \frametitle{Klassifikation politischer Regime nach Przeworski et al.}
  \begin{itemize}
    \item minimaler, prozeduraler Demokratiebegriff
    \item Dichotomie auf Basis individuell notw., gemeinsam hinr. Bdg.
    \item nur objektive Merkmale berücksichtigt
    \item explizite Darstellung systematischer Messfehler
  \end{itemize}

  \begin{quote}
    \normalfont
    \small
    ``Our purpose is to distinguish between (1) regimes that
    allow some, even if limited, regularized competition
    among conflicting visions and interests and (2) regimes
    in which some values or interests enjoy a monopoly
    buttressed by the threat or the actual use of force.
    Thus `democracy', for us, is a regime in which those who
    govern are selected through contested elections. This
    definition has two parts: `government' and
    `contestation'.'' (S. 15)
  \end{quote}
\end{frame}

\begin{frame}
  \frametitle{Konzeptbaum der Klassifikation von Przeworski et al.}
  \begin{figure}[t]
    \begin{adjustbox}{max totalsize={\textwidth}{.9\textheight}, center}
\begin{tikzpicture}[
  every matrix/.append style={ampersand replacement=\&,matrix of nodes},
  edge from parent/.append style={-latex}
]
% --- Layout & Styles --------------------------------------
\tikzstyle{solid node}=[circle,draw,inner sep=1.2,fill=black];
\tikzstyle{hollow node}=[circle,draw,inner sep=1.2];
\tikzstyle{level 1}=[sibling distance = 18em]
\tikzstyle{level 2}=[sibling distance = 7em]
% --- Tree -------------------------------------------------
\node(0){Democracy}
  child{
    node(0-1){Government}
      child{node(0-1-1){Executive}
        child{node{\sc \#1 exselec}}
      }
      child{node(0-1-2){Legislative}
        child{node{\sc \#2 legselec}}
      }
  }
  child{
    node(0-2){Contestation}
    child{ node(0-2-1) [matrix] {\textit{Ex ante} \\ uncertainty \\}
      child{node{\sc \#3 party}}
    }
    child{ node(0-2-2) [matrix] {\textit{Ex post} \\ irreversibility \\}
      child{node{\sc \#4 typeii}}
    }
    child{ node(0-2-3) [matrix] {Repeata- \\ bility \\}
      child{node{\sc \#3 incumb}}
    }
  } ;
\end{tikzpicture}
\end{adjustbox}
  \end{figure}
\end{frame}

\begin{frame}
  \frametitle{{\sc typeii}: Ärger mit dem Regierungswechsel}
  \begin{itemize}
    \item Lässt eine Regierung nur wählen, weil sie nicht verliert?
    \item[$\rightarrow$] Objektive Merkmale (\#1--3) nur notwendig
    \item[$\rightarrow$] Suffizienz durch systematischer Messfehler erkauft
    \item ``Err we must; the question is which way.'' (23)
    \begin{itemize}
      \item Typ 1 Wenn im Zweifel, dann Autokratie.
      \item Typ 2 Wenn im Zweifel, dann Demokratie.
    \end{itemize}
    \item ``We choose to take a cautious stance, that is, to avoid type-II errors.'' (25)
    \item [$\rightarrow$] Wie einflussreich ist diese Regel eigentlich?
  \end{itemize}
\end{frame}

\end{document}