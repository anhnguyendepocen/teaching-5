% Preamble -------------------------------------------------
\documentclass{beamer}
\usepackage[utf8]{inputenc}
\usepackage[ngerman]{babel}
\usepackage{hyperref}
\usepackage{graphicx}
\usepackage{booktabs}
\usepackage{setspace}

% Slides setup ---------------------------------------------
\usetheme{Berlin}
\usecolortheme{seagull}
\usefonttheme{professionalfonts}

\title{Rückmeldung zu den Reaktionspapieren}
\author{Dag Tanneberg\thanks{%
  \href{mailto:dag.tanneberg@uni-potsdam.de}%
    {dag.tanneberg@uni-potsdam.de}
  }
}
\institute[Universität Potsdam]{
  {\glqq}Wie erkl\"art man autorit\"are Herrschaft?{\grqq}\\
  Universität Potsdam\\
  Lehrstuhl für Vergleichende Politikwissenschaft\\
  Sommersemester 2018
}
\date{\today}
% document body --------------------------------------------
\begin{document}

\maketitle

\begin{frame}
\frametitle{Leitfragen der Sitzung}
\begin{enumerate}
  \item Wozu dient ein Reaktionspapier?
  \item Was lieft gut?
  \item Was lief schlecht?
\end{enumerate}
\end{frame}

\begin{frame}
  \frametitle{Wozu dient ein Reaktionspapier?}
  Das Reaktionspapier\dots
  \begin{itemize}
    \item zwingt zu konzentriertem Lesen;
    \item stellt eine Verbindung zwischen Lektüre und Diskussion her;
    \item verlangt die Entwicklung einer eigenständigen Position;
    \item trainiert die pointierte schriftliche Stellungnahme;
    \item protokolliert \textit{nicht}.
  \end{itemize}
\end{frame}

\begin{frame}
  \frametitle{Was lief gut?}
  Die meisten Reaktionspapiere\dots
  \begin{itemize}
    \item ordneten den Beitrag in einen größeren Zusammenhang ein;
    \item formulierten eine abschließende kritische Würdigung;
    \item orientierten sich klar an der Seminardiskussion.
  \end{itemize}
\end{frame}

\begin{frame}
  \frametitle{Was lief schlecht?}
  Etliche Reaktionspapiere\dots
  \begin{itemize}
    \item paraphrasierten lediglich Argumente der Seminardiskussion;
    \item setzten auf Masse statt auf Klasse;
    \item hielten die selbständige Auseinandersetzung kurz;
    \item machten die Implikationen der eigenen Argumente nicht klar.\newline
    \begin{quote}
      \scriptsize
      Die Variable ODWP (other democracies in the world, as a
      percentage) soll den externen Druck auf die Staaten
      abbilden, der steigt, sobald mehr Staaten demokratisch
      strukturiert sind. Daraus entwickeln sie folgende These:
      {\glqq}More democracies throughout the world should
      increase opposition strength and dampen the autocrat’s
      enthusiam to repress, leading to more institutional
      concessions (Gandhi et al. 2007: 1286).{\grqq} [\dots] Je nachdem wie man
      Demokratie definieren möchte, lassen sich
      unterschiedliche Annahmen über die Verbreitung von
      demokratischen Staaten treffen. [\dots] An dieser Stelle
      müsste eine weitreichendere Betrachtung zur Anwendung
      kommen, da es keinesfalls zutrifft, dass der Wille von
      Autokraten ihre eigene Bevölkerung zu unterdrücken in
      allen beobachteten Staaten zurückgegangen ist.
    \end{quote}
  \end{itemize}
\end{frame}
\end{document}