% Preamble -------------------------------------------------
\documentclass{beamer}
\usepackage[utf8]{inputenc}
\usepackage[ngerman]{babel}
\usepackage{hyperref}
\usepackage{graphicx}
\usepackage{booktabs}

% Slides setup ---------------------------------------------
\usetheme{Berlin}
\usecolortheme{seagull}
\usefonttheme{professionalfonts}

\title{Rückmeldung zu den Reaktionspapieren}
\author{Dag Tanneberg\thanks{%
  \href{mailto:dag.tanneberg@uni-potsdam.de}%
    {dag.tanneberg@uni-potsdam.de}
  }
}
\institute[Universität Potsdam]{
  {\glqq}Wie erkl\"art man autorit\"are Herrschaft?{\grqq}\\
  Universität Potsdam\\
  Lehrstuhl für Vergleichende Politikwissenschaft\\
  Sommersemester 2018
}
\date{\today}
% document body --------------------------------------------
\begin{document}

\maketitle

\begin{frame}
\frametitle{Leitfragen der Sitzung}
\begin{enumerate}
  \item Wozu dient ein Reaktionspapier?
  \item Was lieft gut?
  \item Was lief schlecht?
\end{enumerate}
\end{frame}

\begin{frame}
  \frametitle{Wozu dient ein Reaktionspapier?}
  Das Reaktionspapier\dots
  \begin{itemize}
    \item zwingt zu konzentriertem Lesen;
    \item stellt eine Verbindung zwischen Lektüre und Diskussion her;
    \item verlangt die Entwicklung einer eigenständigen Position;
    \item trainiert die pointierte schriftliche Stellungnahme;
    \item protokolliert nicht.
  \end{itemize}
\end{frame}

\begin{frame}
  \frametitle{Was lief gut?}

\end{frame}

\begin{frame}
  \frametitle{Was lief schlecht?}
  \begin{itemize}
    \item Umfang und Qualität verfügbarer Information
    \item Bulgarien ($\ge 1953$)
    \begin{itemize}
      \item gezielte Rekrutierung selbstinteressierter Informanten
      \item fortlaufende Qualitätskontrolle
      \item polit. Repression wird selektiv \& präventiv
    \end{itemize}
    \item Irak
    \begin{itemize}
      \item breite (Zwangs-)Rekrutierung von Informanten
      \item mangelnge Qualitätskontrolle (evt. Überlastung?)
      \item polit. Repression bleibt willkürlich
    \end{itemize}
  \end{itemize}
\end{frame}

\end{document}