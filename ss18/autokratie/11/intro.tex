% Preamble -------------------------------------------------
\documentclass{beamer}
\usepackage[utf8]{inputenc}
\usepackage[ngerman]{babel}
\usepackage{hyperref}
\usepackage{graphicx}

% Slides setup ---------------------------------------------
\usetheme{Berlin}
\usecolortheme{seagull}
\usefonttheme{professionalfonts}

\title{Zusammenfassung vom 25.06.2018}
\author{Dag Tanneberg\thanks{%
  \href{mailto:dag.tanneberg@uni-potsdam.de}%
    {dag.tanneberg@uni-potsdam.de}
  }
}
\institute[Universität Potsdam]{
  {\glqq}Wie erkl\"art man autorit\"are Herrschaft?{\grqq}\\
  Universität Potsdam\\
  Lehrstuhl für Vergleichende Politikwissenschaft\\
  Sommersemester 2018
}
\date{02.07.2018}
% document body --------------------------------------------
\begin{document}

\maketitle

\begin{frame}
  \frametitle{Leitfragen der Sitzung}
  \begin{enumerate}
    \item Warum stellt soziale Ungleichheit ein Problem dar?
    \item Was erklärt \textit{ceteris paribus} divergierende Ungleichheitsgrade?
    \item Designprobleme
  \end{enumerate}
\end{frame}

\begin{frame}
  \frametitle{Warum stellt soziale Ungleichheit ein Problem dar?}
  \begin{enumerate}
    \item Problem autoritärer Kontrolle
    \begin{itemize}
      \item Legitimationsansprüche autoritärer Eliten einlösen
      \item Revolutionsversuchen vorbeugen
    \end{itemize}
    \item Problem autoritärer Machtteilung
    \begin{itemize}
      \item Unterstützerbasis des Regimes verbreitern
      \item
    \end{itemize}
  \end{enumerate}
\end{frame}

\begin{frame}
  \frametitle{Was erklärt \textit{cet. par.} divergierende Ungleichheitsgrade?}
  \begin{itemize}
    \item Ergebnis pfadabh. Entw. nach kritischer Weichenstellung
  \end{itemize}
  \begin{enumerate}
    \item Vietnam nach 1989
    \begin{itemize}
      \item Waffengleichheit zw. 3 rivalisierenden Gruppen
      \item Aufwertung des ZK ggü. Politbüro \& Ständigem Ausschuss
      \item steigender Wettbewerb um Mitgliedschaft im ZK
      \item [$\rightarrow$] Machtdiffusion nötigt zur Bildung breiter Koalitionen
    \end{itemize}
    \item China nach Tiananmen
    \begin{itemize}
      \item Dominanz des Ständigen Ausschuss des ZK
      \item ZK lediglich symbol. Rolle des ZK bei Führungswechsel
      \item Parteistatute restringieren polit. Wettbewerb um ZK-Posten
      \item [$\rightarrow$] Machtkonzentration ermöglicht die Bildung enger Koalitionen
    \end{itemize}
  \end{enumerate}
\end{frame}

\begin{frame}
  \frametitle{Designprobleme}
  \begin{enumerate}
    \item externe Validität
    \begin{itemize}
      \item Vergleichende Fallstudie kommunistischer Einparteisstaaten
      \item Wie verallgemeinerbar sind die Ergebnisse?
    \end{itemize}
    \item interne Validität
    \begin{itemize}
      \item Bindewirkung politischer Institutionen zweifelhaft
      \item Henne-Ei-Problem von Akteursinteressen und Institutionen
    \end{itemize}
      \item Leverage
    \begin{itemize}
      \item Welche weiteren Implikationen hat die Theorie?
    \end{itemize}
  \end{enumerate}
\end{frame}

\end{document}