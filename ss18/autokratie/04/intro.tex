% Preamble -------------------------------------------------
\documentclass{beamer}
\usepackage[utf8]{inputenc}
\usepackage[ngerman]{babel}
\usepackage{hyperref}
\usepackage{graphicx}
\usepackage{tikz}
  \usetikzlibrary{positioning, arrows, matrix, calc, backgrounds, fit, decorations.pathmorphing}
\usepackage{adjustbox}
\definecolor{grey538}{RGB}{240,240,240}
\usepackage{subcaption}
\usepackage[%
  isbn = false, doi = false, url = false
]{biblatex}
\usepackage{bibentry}
\usepackage{amsmath}
\addbibresource{./library.bib}

% Slides setup ---------------------------------------------
\usetheme{Berlin}
\usecolortheme{seagull}
\usefonttheme{professionalfonts}

\title{Zusammenfassung vom 30.04.2018}
\author{Dag Tanneberg\thanks{%
  \href{mailto:dag.tanneberg@uni-potsdam.de}%
    {dag.tanneberg@uni-potsdam.de}
  }
}
\institute[Universität Potsdam]{
  {\glqq}Wie erkl\"art man autorit\"are Herrschaft?{\grqq}\\
  Universität Potsdam\\
  Lehrstuhl für Vergleichende Politikwissenschaft\\
  Sommersemester 2018
}
\date{07.05.2018}
% document body --------------------------------------------
\begin{document}

\maketitle

\begin{frame}
  \frametitle{Leitfragen der Sitzung}
  \begin{enumerate}
    \item Warum haben Daten über autoritäre Herrschaft mindere Qualität?
    \item Was steht auf dem Spiel?
    \item Was können wir tun?
  \end{enumerate}
\end{frame}

\begin{frame}
  \frametitle{Warum haben Daten über a.\,H. mindere Qualität?}
  \begin{columns}
    \begin{column}{.5\textwidth}
      \begin{figure}[t]
      \centering
      \textbf{Wichtige Qualitätskriterien}
      \begin{align*} % force horizontal alignment of minipages
        \begin{minipage}{.49\textwidth}
        \centering
          \begin{tikzpicture}[
            grow = right, edge from parent/.style={draw,-latex},
            level distance = 8em, sibling distance = 8em, minimum size = 2em, inner sep = 0,
            scale = 1.5
          ]
            \tikzstyle {latent} = [draw, shape = circle, fill = white]
            \tikzstyle {observed} = [draw, shape = rectangle, fill = white]
            % place nodes
            \path node (0) [latent] at (0,0) {$\theta$};
            \path node (1) [observed] at (1,0) {$x$};
            \path node (2) [latent] at (1,1) {$\epsilon$};
            % place edges
            \foreach \x in {0, 2}
              { \draw [-latex] (\x) -- (1) ; } ;
            % fill background area
            \begin{scope}[on background layer]
              \node [fill=grey538, fit = (0) (1) (2)] {};
            \end{scope}
          \end{tikzpicture}
          \subcaption{Reliabilität}
        \end{minipage}
        \begin{minipage}{.5\textwidth}
          \centering
          \begin{tikzpicture}[
            grow = right, edge from parent/.style={draw,-latex},
            level distance = 8em, sibling distance = 8em, minimum size = 2em, inner sep = 0,
            scale = 1.5
          ]
            \tikzstyle {latent} = [draw, shape = circle, fill = white]
            \tikzstyle {observed} = [draw, shape = rectangle, fill = white]
            % place nodes
            \path node (0) [latent] at (0,0) {$\theta$};
            \path node (1) [observed] at (1,0) {$x$};
            \path node (2) [observed, inner sep = 1] at (.5,1) {Drittvariable};
            % place edges
            \foreach \x in {0, 1}
              { \draw [-latex] (2) -- (\x) ; } ;
            \draw [-latex, ] (0) [dashed] to (1) ;
            \begin{scope}[on background layer]
              \node [fill=grey538, fit = (0) (1) (2)] {};
            \end{scope}
          \end{tikzpicture}
          \subcaption{Validität}
        \end{minipage}
      \end{align*}
      \end{figure}
      % \begin{itemize}
      %   \item Zuverlässigkeit \& Gültigkeit
      %   \item (a) bedingt (b)
      % \end{itemize}
    \end{column}
    \hfill
    \begin{column}{.49\textwidth}
      \textbf{Ursachen minderer Qualität}
      \begin{enumerate}
        \item strategische Intransparenz:\\Will man berichten?
        \item Kapazitätsprobleme:\\Kann man berichten?
        \item Zugang:\\Darf man berichten?
      \end{enumerate}
    \end{column}
  \end{columns}
\end{frame}

\begin{frame}
  \frametitle{Was steht auf dem Spiel?}
  \begin{columns}[t]
    \begin{column}{.5\textwidth}
      \textbf{Gültigkeit von Theorien}
      \begin{quote}
        \small
        The central problem is that scholars often do not have
        the type of data that would allow them to test causal
        mechanisms that rely on institutional dynamics, elite
        attitudes, or other phenomena that are not readily
        observable. (978)
      \end{quote}
    \end{column}
    \begin{column}{.5\textwidth}
      \textbf{Verlass auf Politikberatung}
      \begin{quote}
        \small
        To my knowledge, comparativists have not looked at the
        Iraq wars from an agnotological perspective and
        investigated the contribution of academics [\dots]
        to the series of dramatic miscalculations that have
        characterized U.S. policy toward Iraq. (979)
      \end{quote}
    \end{column}
  \end{columns}
\end{frame}

\begin{frame}
  \frametitle{Was können wir tun?}
  \begin{columns}
    \begin{column}{.5\textwidth}
      \textbf{Mehr Feldforschung}
    \end{column}
    \begin{column}{.5\textwidth}
      \textbf{Mehr Archivforschung}
    \end{column}
  \end{columns}
\end{frame}

\end{document}