% Preamble -------------------------------------------------
\documentclass{beamer}
\usepackage[utf8]{inputenc}
\usepackage[ngerman]{babel}
\usepackage{hyperref}
\usepackage{graphicx}
\usepackage{tikz}
  \usetikzlibrary{positioning, arrows, matrix, calc, backgrounds, fit, decorations.pathmorphing}
\usepackage{adjustbox}
\definecolor{grey538}{RGB}{240,240,240}
\usepackage{subcaption}
\usepackage[%
  isbn = false, doi = false, url = false
]{biblatex}
\usepackage{bibentry}
\usepackage{amsmath}
\usepackage{booktabs}
\usepackage{multirow}
\addbibresource{./library.bib}

% Slides setup ---------------------------------------------
\usetheme{Berlin}
\usecolortheme{seagull}
\usefonttheme{professionalfonts}

\title{Zusammenfassung vom 04.06.2018}
\author{Dag Tanneberg\thanks{%
  \href{mailto:dag.tanneberg@uni-potsdam.de}%
    {dag.tanneberg@uni-potsdam.de}
  }
}
\institute[Universität Potsdam]{
  {\glqq}Wie erkl\"art man autorit\"are Herrschaft?{\grqq}\\
  Universität Potsdam\\
  Lehrstuhl für Vergleichende Politikwissenschaft\\
  Sommersemester 2018
}
\date{11.06.2018}
% document body --------------------------------------------
\begin{document}

\maketitle

\begin{frame}
  \frametitle{Leitfragen der Sitzung}
  \begin{enumerate}
    \item Warum spricht Smith über autoritäre Regierungsparteien?
    \item Welche Vorteile generieren autoritäre Regierungsparteien?
    \item Unter welchen Voraussetzungen greifen diese Vorteile?
    \item Nachtrag zum Forschungsdesign
  \end{enumerate}
\end{frame}

\begin{frame}
  \frametitle{Warum spricht Smith über autoritäre Regierungsparteien?}
  \begin{itemize}
    \item Lücke in der theoretischen Literatur \newline
      \small{
        ``Missing from the study of authoritarianism is a causal
        account linking origins to institutions and institutions to
        outcomes, [\dots].'' (421)
      }
    \item Charakter autoritärer Regierungsparteien nicht selbstverständlich \newline
    \small{
      ``Treating party institutions as prior variables makes it
      nearly impossible to figure out how incentives within such
      regimes might come, or not come, to look very much like
      the Stag Hunt game.'' (427)
    }
    \item Begleitumstände der Konsolidierung stiften Pfadabhängigkeiten \newline
    \small{
      ``The theory suggests that prospects for long-term
      survival of single-party regimes are best
      conceptualized as a function of the challenges
      political actors face as they make decisions about
      building parties.'' (430)
    }
  \end{itemize}
\end{frame}

\begin{frame}
  \frametitle{Welche Vorteile generieren autoritäre Regierungsparteien?}
  \begin{enumerate}
    \item Mit Blick auf \textit{authoritarian power-sharing}
    \begin{itemize}
      \item Koalitionsmanagement, d.\,h. Streitschlichtungsprozeduren
    \end{itemize}
    \item Mit Blick auf \textit{authoritarian control}
    \begin{itemize}
      \item Repression: Mobilisierung der eigenen Anhängerschaft
      \item Kooptation: Staatsparteien; selektive Gewähr von Vorteilen
    \end{itemize}
  \end{enumerate}
\end{frame}

\begin{frame}
  \frametitle{Unter welchen Voraussetzungen greifen diese Vorteile?}
  \begin{quote}
  ``Elites that face organized opposition in the form of
  highly institutionalized social groups [\dots], \textbf{and}
  that have little or no access to rent sources are
  likely to respond to these constraints
  by building party institutions to mobilize their own
  constituencies.'' (422; my emph.)
\end{quote}
\begin{tabular}{*{3}{c}}
~ & \multicolumn{2}{c}{Organiz. strength of opposition} \\
Rent revenues & High & Low \\ \midrule
Low &  \textbf{Organizat. weapons} \\
~ & Indonesia (1967-71) \\
~ & Tanzania (1954-62) \\
High & ~ & \textbf{Rent havens} \\
~ & ~ & Guinea-Bissau (70-74) \\
~ & ~ & Philippines (1978-82) \\
\end{tabular}
\end{frame}

\begin{frame}
  \frametitle{Nachtrag zum Forschungsdesign}
  \begin{itemize}
    \item Mixed Methods Approach
    \item Quantitative Analyse: Effekt nicht robust
    \begin{itemize}
      \item Werden die Koeffizienten korrekt interpretiert?
      \item Wie belastbar sind Aussagen über statistische Signifikanz?
      \item Ausschluss von Mexiko \& UdSSR: Selection bias?
    \end{itemize}
    \item Qualitative Analyse: Konsolidierungsbdg. plausibilisieren
    \begin{itemize}
      \item Kombination von Method of Agreement \& Difference
      \item Wie aussagekräftig ist die Analyse von Krisen?
      \item Bestimmt der Zeitraum der Analyse auf das Ergebnis?
    \end{itemize}
  \end{itemize}
\end{frame}

\end{document}