% Preamble -------------------------------------------------
\documentclass{beamer}
\usepackage[utf8]{inputenc}
\usepackage[ngerman]{babel}
\usepackage{hyperref}
\usepackage{graphicx}
\usepackage{booktabs}

% Slides setup ---------------------------------------------
\usetheme{Berlin}
\usecolortheme{seagull}
\usefonttheme{professionalfonts}

\title{Zusammenfassung vom 02.07.2018}
\author{Dag Tanneberg\thanks{%
  \href{mailto:dag.tanneberg@uni-potsdam.de}%
    {dag.tanneberg@uni-potsdam.de}
  }
}
\institute[Universität Potsdam]{
  {\glqq}Wie erkl\"art man autorit\"are Herrschaft?{\grqq}\\
  Universität Potsdam\\
  Lehrstuhl für Vergleichende Politikwissenschaft\\
  Sommersemester 2018
}
\date{09.07.2018}
% document body --------------------------------------------
\begin{document}

\maketitle

\begin{frame}
  \frametitle{Leitfragen der Sitzung}
  \begin{enumerate}
    \item Was ist polit. Repression, warum ist sie wichtig?
    \item In welchen Varianten wird polit. Repression ausgeübt?
    \item Was beeinflusst die Ausübung polit. Repression?
  \end{enumerate}
\end{frame}

\begin{frame}
  \frametitle{Was ist polit. Repression, warum ist sie wichtig?}
  \begin{itemize}
    \item \textbf{Definition}: ``[\dots] any action by another group which raises the contender's cost of collective action. (We call repression [\dots] political if the other party is a government.)''\newline
    \begin{footnotesize}(Tilly (1978): From Mobilization to Revolution. Newbery Award Records, Inc.: New York,  100)\end{footnotesize}
  \item \textbf{Relevanz}:``[N]o dictatorship can do away with repression. The
    lack of popular consent -- inherent in any political
    system where a few govern over the many -- is the
    `original sin' of dictatorships.'' (Svolik 2012: 10)
  \end{itemize}
\end{frame}

\begin{frame}
  \frametitle{In welchen Varianten wird polit. Repression ausgeübt?}
  \begin{table}
    \centering
    \begin{tabular}{l*{2}{c}}
      \toprule
      ~ & \multicolumn{2}{c}{Zielbestimmung} \\
      Zeitpunkt & Selektiv & Willkürlich \\
      \cmidrule{2-3}
      Präventiv & Bulgarian ($\ge 1953$) & Irak; Bulgarian ($< 1953$)\\
      Reaktiv & & Irak; Bulgarian ($< 1953$)\\
      \bottomrule
    \end{tabular}
  \end{table}
\end{frame}

\begin{frame}
  \frametitle{Was beeinflusst die Ausübung polit. Repression?}
  \begin{itemize}
    \item Umfang und Qualität verfügbarer Information
    \item Bulgarien ($\ge 1953$)
    \begin{itemize}
      \item gezielte Rekrutierung selbstinteressierter Informanten
      \item fortlaufende Qualitätskontrolle
      \item polit. Repression wird selektiv \& präventiv
    \end{itemize}
    \item Irak
    \begin{itemize}
      \item breite (Zwangs-)Rekrutierung von Informanten
      \item mangelnge Qualitätskontrolle (evt. Überlastung?)
      \item polit. Repression bleibt willkürlich
    \end{itemize}
  \end{itemize}
\end{frame}

\end{document}