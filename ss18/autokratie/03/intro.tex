% Preamble -------------------------------------------------
\documentclass{beamer}
\usepackage[utf8]{inputenc}
\usepackage[ngerman]{babel}
\usepackage{hyperref}
\usepackage{graphicx}
\usepackage{tikz}
  \usetikzlibrary{positioning}
  \usetikzlibrary{calc}
  \usetikzlibrary{matrix}
\usepackage{adjustbox}

\usepackage[%
  isbn = false, doi = false, url = false
]{biblatex}
\usepackage{bibentry}
\usepackage{amsmath}
\addbibresource{./library.bib}

% Slides setup ---------------------------------------------
\usetheme{Berlin}
\usecolortheme{seagull}
\usefonttheme{professionalfonts}

\title{Zusammenfassung vom 16.04.2018}
\author{Dag Tanneberg\thanks{%
  \href{mailto:dag.tanneberg@uni-potsdam.de}%
    {dag.tanneberg@uni-potsdam.de}
  }
}
\institute[Universität Potsdam]{
  {\glqq}Wie erkl\"art man autorit\"are Herrschaft?{\grqq}\\
  Universität Potsdam\\
  Lehrstuhl für Vergleichende Politikwissenschaft\\
  Sommersemester 2018
}
\date{30.04.2018}
% document body --------------------------------------------
\begin{document}

\maketitle

\begin{frame}
  \frametitle{Leitfragen der Sitzung}
  \begin{enumerate}
    \item Welches Spektrum von Regierungsformen umfasst
      autoritäre Herrschaft?
    \item Wie verändern sich die Ziele autoritärer
      Herrschaft, wenn sie sich der Demokratie angleicht?
    \item Was unterscheidet den Totalitarismus von anderen
      Formen autoritärer Herrschaft?
  \end{enumerate}
\end{frame}

\begin{frame}
  \frametitle{Definition des Autoritarismus nach Juan J. Linz}
  \begin{quote}
    \normalfont
    \small
    ``political systems with \textbf{limited, not responsible,
    political pluralism}, without elaborate and guiding
    ideology, but with \textbf{distinctive mentalities},
    \textbf{without extensive nor intensive political mobilization},
    except at some points in their development, and in which
    a leader or occasionally a small group exercises power
    within formally ill-defined limits but actually quite
    predictable ones''
    \newline \newline
    \footnotesize{\fullcite[159]{Linz.2000}}.
  \end{quote}
  \begin{itemize}
    \item autorit. Herrschaft $\equiv \{\text{Totalitarismus; Demokratie}\}^\complement$
    \item mehrdimensionale, graduelle Unterscheidung
    \item variabler Herrschaftsanspruch:

    Kontrolle $\leftrightarrow$ Entpolitisierung $\leftrightarrow$ Voluntarismus
  \end{itemize}
\end{frame}

\begin{frame}
  \frametitle{Konzeptbaum des Ansatzes von Juan Linz}
  \begin{figure}[t]
    \begin{adjustbox}{max totalsize={\textwidth}{.9\textheight}, center}
\begin{tikzpicture}[
  every matrix/.append style={ampersand replacement=\&,matrix of nodes},
  edge from parent/.append style={-latex}
]
% --- Layout & Styles --------------------------------------
\tikzstyle{solid node}=[circle,draw,inner sep=1.2,fill=black];
\tikzstyle{hollow node}=[circle,draw,inner sep=1.2];
\tikzstyle{level 1}=[sibling distance = 12em]
\tikzstyle{level 2}=[sibling distance = 6em]
% --- Tree -------------------------------------------------
\node(0){Politisches Regime}
  child{
    node(0-1){Pluralismus}
    child{
      node(0-1-1){Totalitär}
    }
    child{
      node(0-1-2){Demokratisch}
    }
  }
  child{
    node(0-2){Ideologie}
    child{
      node(0-2-1){Totalitär}
    }
    child{
      node(0-2-2){Demokratisch}
    }
  } 
  child{
    node(0-3){Mobilisierung}
    child{
      node(0-3-1){Totalitär}
    }
    child{
      node(0-3-2){Demokratisch}
    }
  } ;
\coordinate (arc-1-1) at ($(0-1)!.5!(0-1-1)$) ;
\coordinate (arc-1-2) at ($(0-1)!.5!(0-1-2)$) ;
\draw [-] (arc-1-1) to [bend right] (arc-1-2) ;
\coordinate (arc-2-1) at ($(0-2)!.5!(0-2-1)$) ;
\coordinate (arc-2-2) at ($(0-2)!.5!(0-2-2)$) ;
\draw [-] (arc-2-1) to [bend right] (arc-2-2) ;
\coordinate (arc-3-1) at ($(0-3)!.5!(0-3-1)$) ;
\coordinate (arc-3-2) at ($(0-3)!.5!(0-3-2)$) ;
\draw [-] (arc-3-1) to [bend right] (arc-3-2) ;
\end{tikzpicture}
\end{adjustbox}
  \end{figure}
\end{frame}

\begin{frame}
  \frametitle{Kritische Anmerkungen}
  \begin{enumerate}
    \item autoritäre Herrschaft \textit{ex negativo} bestimmt
    \item Wechselbeziehung zw. Merkmalen wenig elaboriert
    \item Annahmen über Ziele kaum theoretisch geerdet
    \item implizite Schwellwerte für Klassifikationen (Compet. Authorit.)
  \end{enumerate}
\end{frame}

\end{document}