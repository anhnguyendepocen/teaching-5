% Preamble -------------------------------------------------
\documentclass{beamer}
\usepackage[utf8]{inputenc}
\usepackage[ngerman]{babel}
\usepackage{adjustbox}
\usepackage{tikz}
  \usetikzlibrary{positioning, calc, decorations.pathreplacing, backgrounds, fit}
\usepackage{multirow}
\usepackage{graphicx}
\usepackage{caption}
\usepackage{subcaption}
\usepackage{booktabs}

\definecolor{grey538}{rgb}{240,240,240}

% Slides setup ---------------------------------------------
\usetheme{Berlin}
\usecolortheme{seagull}
\usefonttheme{professionalfonts}

\title{Zusammenfassung vom 22. Januar 2018}
\author{Dag Tanneberg\thanks{%
  \href{mailto:dag.tanneberg@uni-potsdam.de}%
    {dag.tanneberg@uni-potsdam.de}
  }
}
\institute[Universität Potsdam]{
  {\glqq}Forschungsdesign in den Sozialwissenschaften{\grqq}\\
  Universität Potsdam\\
  Lehrstuhl für Vergleichende Politikwissenschaft\\
  Wintersemester 2017/2018
}
\date{29. Januar 2018}

\begin{document}
\maketitle

\begin{frame}
\frametitle{Leitfragen}
\begin{enumerate}
  \item Für welche Variablen muss ich kontrollieren?
  \item Wie bewältige ich Endogenität?
\end{enumerate}
\end{frame}

\begin{frame}
\frametitle{Auf welche Variablen muss ich kontrollieren?}
\begin{enumerate}
  \item In den amerikanischen Bundesstaaten nimmt die
    Scheidungsrate zu, wenn die Hochzeitsrate steigt. Das
    Alter der Eheschließenden spielt eine erheblich Rolle.
    Junge Menschen heiraten häufiger und sie lassen sich
    öfter scheiden.
  \item Nicht-demokratische Einparteistaaten weisen häufig
    einen höheren sozialen Entwicklungsstand als andere
    autoritäre Regime auf. Mit einer kommunistischen
    Ideologie geht sowohl das Herrschaftsmonopol einer
    einzigen Partei als auch ein Bemühen um soziale
    Entwicklung einher.
\end{enumerate}
\end{frame}

\begin{frame}
\frametitle{Für welche Variablen muss ich kontrollieren?}
\textbf{Störgröße, Drittvariable, Kontrollvariable, \dots}
\begin{itemize}
  \item Variable, die die kausale Beziehung $X \implies Y$ beeinflusst
  \item [$\rightarrow$] kausaler Effekt von $X$ wird aufgebläht oder versteckt
\end{itemize}

\begin{figure}
\begin{subfigure}{.3\textwidth}
  \centering
  \subcaption{Irrelevant}
  \begin{tikzpicture}
    \node (Z) at (.5, 0) {Z} ;
    \node (X) at (0, 1) {X} ;
    \node (Y) at (1, 1) {Y} ;
    \draw [-latex] (X) to (Y) ;
    \draw [-latex] (Z) to (X) ;
  \end{tikzpicture}
\end{subfigure}
  \hfill
\begin{subfigure}{.3\textwidth}
  \centering
  \subcaption{altern. Ursache}
  \begin{tikzpicture}
    \node (Z) at (.5, 0) {Z} ;
    \node (X) at (0, 1) {X} ;
    \node (Y) at (1, 1) {Y} ;
    \draw [-latex] (X) to (Y) ;
    \draw [-latex] (Z) to (Y) ;
  \end{tikzpicture}
\end{subfigure}
\hfill
\begin{subfigure}{.3\textwidth}
  \centering
  \subcaption{Störgröße}
  \begin{tikzpicture}
    \node (Z) at (.5, 0) {Z} ;
    \node (X) at (0, 1) {X} ;
    \node (Y) at (1, 1) {Y} ;
    \draw [-latex, dashed] (X) -- (Y) ;
    \draw [-latex] (Z) to (X) ;
    \draw [-latex] (Z) to (Y) ;
  \end{tikzpicture}
\end{subfigure}
\end{figure}
\textbf{Wie berücksichtige ich Störgrößen?}
\begin{itemize}
  \item Statistische Analyse: Prädiktor aufnehmen
  \item Qualitatitve Analyse: z.\,B. bei der Fallauswahl
\end{itemize}
\end{frame}

\begin{frame}
\frametitle{Für welche Variablen muss ich kontrollieren?}
\textbf{Störgrößen sorgfältig modellieren, denn \dots}
\begin{enumerate}
  \item Ineffiziente Erklärung: Argumentation mind. nicht sparsam
  \item Post-treatment bias: \tikz{%
    \node (X) at (0, 0) {X} ;%
    \node (Xprime) at (1, 0) {X'} ;%
    \node (Y) at (2, 0) {Y} ;%
    \draw [-latex] (X) -- (Xprime);%
    \draw [-latex] (Xprime) -- (Y) ;%
  }
\end{enumerate}

Kommunistische autoritäre Regime weisen häufig
einen höheren sozialen Entwicklungsstand als andere
autoritäre Regime auf. Eine kommunistische
Ideologie begründet in der Regel das Herrschaftsmonopol
einer einzigen Partei. Wenn die Fallauswahl sich auf
Einparteiregime beschränkt, dann verzerrt sie den
Effekt einer kommunistischen Ideologie auf den sozialen
Entwicklungsstand.
\end{frame}

\begin{frame}
\frametitle{Wie bewältige ich Endogenität?}

\textbf{Endogenität}
\begin{itemize}
  \item erkl. Var. X nicht Ursache, sondern Folge der abh. Var. Y
\end{itemize}

\textbf{Beispiel}
\begin{figure}
  \centering
  \begin{tikzpicture}
  \node (Educ) {Bildung};
  \node (Tol) [right = of Educ] {Toleranz} ;
  \node (Gay) [right = of Tol] {\begin{tabular}{c} Befürwortung \\ gleichgeschl. Ehe \end{tabular}} ;
  \draw (Educ) -- (Tol) -- (Gay) ;
  \draw node at ($(Educ)!.5!(Tol)$) [label = north:?] {} ;
  \draw node at ($(Tol)!.375!(Gay)$) [label = north:?] {} ;
  \end{tikzpicture}
\end{figure}

\textbf{Endogenität droht immer, wenn\dots}
\begin{itemize}
  \item keine Kontrolle über Experimental- und Kontrollgruppe
  \item [$\rightarrow$] betrifft v.\,a. nicht-experimentelle Beobachtungsdaten
\end{itemize}
\end{frame}

\begin{frame}
\frametitle{Endogenität droht immer, wenn\dots}

\textbf{Lösungsstrategien}
\begin{enumerate}
  \item Abhängige bzw. unabhängige Variable enger fassen
  \item [$\rightarrow$] bestimmte Arten/Aspekte von Bildung/Toleranz betroffen?
  \item Endogenität in eine Störgröße überführen
  \item [$\rightarrow$] auf den Wohnort kontrollieren
  \item Fälle mit Bedacht auswählen
  \item [$\rightarrow$] Gibt es deviante Fälle?
\end{enumerate}
\end{frame}


\end{document}