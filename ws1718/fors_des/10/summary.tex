% Preamble -------------------------------------------------
\documentclass{beamer}
\usepackage[utf8]{inputenc}
\usepackage[ngerman]{babel}
\usepackage{adjustbox}
\usepackage{tikz}
  \usetikzlibrary{positioning, calc, decorations.pathreplacing, backgrounds, fit}
\usepackage{multirow}
\usepackage{graphicx}
\usepackage{caption}

\definecolor{grey538}{rgb}{240,240,240}

% Slides setup ---------------------------------------------
\usetheme{Berlin}
\usecolortheme{seagull}
\usefonttheme{professionalfonts}

\title{Zusammenfassung vom 18. Dezember 2017}
\author{Dag Tanneberg\thanks{%
  \href{mailto:dag.tanneberg@uni-potsdam.de}%
    {dag.tanneberg@uni-potsdam.de}
  }
}
\institute[Universität Potsdam]{
  {\glqq}Grundlagen der Vergleichenden Politikwissenschaft{\grqq}\\
  Universität Potsdam\\
  Lehrstuhl für Vergleichende Politikwissenschaft\\
  Wintersemester 2017/2018
}
\date{8. Januar 2018}

\begin{document}
\maketitle

\begin{frame}
  \frametitle{Leitfragen}
  \begin{enumerate}
    \item Was sind Konzepte?
    \item Wie baut man Konzepte auf?
    \item Welche Fallstricke muss man beachten?
  \end{enumerate}
\end{frame}

\begin{frame}

\textbf{Konzepte sind}
  \begin{enumerate}
    \item allgemeine Aussagen $\rightarrow$ Klassen von Phänomenen
    \item abstrakte Repräsenationen $\rightarrow$ nicht sensorisch erfahrbar
    \item Bausteine von Theorien $\rightarrow$ betreffen x oder y, aber nicht x $\sim$ y
  \end{enumerate}
\vfill
  \begin{columns}
    \begin{column}{.5\textwidth}
    \textbf{(a) Klassische Konzepte}
    \begin{itemize}
      \item formulieren Merkmale
      \item regelbasierte Hierarchie
      \item notw. \& hinr. Bdg.
    \end{itemize}
    \end{column}
    \begin{column}{.5\textwidth}
    \textbf{(b) Prototypische Konzepte}
    \begin{itemize}
      \item formulieren Merkmale
      \item Verteilung statt Hierarchie
      \item Familienähnlichkeiten
    \end{itemize}
    \end{column}
  \end{columns}
\end{frame}

\begin{frame}
\frametitle{Wie baut man Konzepte auf?}
\begin{columns}
\begin{column}{.4\textwidth}
\centering
\begin{tikzpicture}[
  edge from parent/.style = {draw, -latex},
  scale = 1
]
\tikzstyle{every node} = [fill, circle, inner sep = .1em]
\node (root) {}
  child{ node (1) {}
    child{ node (1-1) {} }
    child{ node (1-2) {} }
    child{ node (1-3) {} }
  }
  child{ node (2) {} }
  child{ node (3) {} };
% \node (4) [right = of root, fill = none] {Begriff} ;
% \node (5) [right = of 2, fill = none] {Merkmale} ;
% \node (6) [right = of 1-3, fill = none] {Indikatoren} ;
\begin{scope}[on background layer]
  \node [fill = gray, opacity = .25, shape = rectangle, fit = (root) (1-1) (3)] {};
\end{scope}
\end{tikzpicture}
\end{column}
\begin{column}{.6\textwidth}
\textbf{Drei Ebenen der Konzeptbildung}
\begin{enumerate}
  \item Begriff
  \begin{itemize}
    \item Bezeichnung des K.
    \item unsytematisiertes Hintergrundk.
  \end{itemize}
  \item Merkmale
  \begin{itemize}
    \item bestimmen Intension des K.
    \item systematisiertes K.
  \end{itemize}
  \item Indikatoren
  \begin{itemize}
    \item Messbarmachung des K.
    \item Übergang zur Extension
  \end{itemize}
\end{enumerate}
\end{column}
\end{columns}
\end{frame}

\begin{frame}
\frametitle{Welche Fallstricke muss man beachten?}
\begin{itemize}
    \item Profusion, d.i. Innovation ohne Notwendigkeit
    \item Stretching, d.i. Anwendung auf nicht zugehörige Fälle
    \item Funktionale Äquivalente, d.i. Phänomene mit gleichem Effekt
    \item Redundanz, d.i. Doppelung von Merkmalen
\end{itemize}
\end{frame}
\end{document}