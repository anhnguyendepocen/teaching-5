% Preamble -------------------------------------------------
\documentclass{beamer}
\usepackage[utf8]{inputenc}
\usepackage[ngerman]{babel}
\usepackage{adjustbox}
\usepackage{tikz}
  \usetikzlibrary{positioning, calc, decorations.pathreplacing, backgrounds, fit}
\usepackage{multirow}
\usepackage{graphicx}
\usepackage{caption}

\definecolor{grey538}{rgb}{240,240,240}

% Slides setup ---------------------------------------------
\usetheme{Berlin}
\usecolortheme{seagull}
\usefonttheme{professionalfonts}

\title{Zusammenfassung vom 5. Januar 2018}
\author{Dag Tanneberg\thanks{%
  \href{mailto:dag.tanneberg@uni-potsdam.de}%
    {dag.tanneberg@uni-potsdam.de}
  }
}
\institute[Universität Potsdam]{
  {\glqq}Forschungsdesign in den Sozialwissenschaften{\grqq}\\
  Universität Potsdam\\
  Lehrstuhl für Vergleichende Politikwissenschaft\\
  Wintersemester 2017/2018
}
\date{15. Januar 2018}

\begin{document}
\maketitle

\begin{frame}
  \frametitle{Leitfragen}
  \begin{enumerate}
    \item Warum ist die Fallauswahl wichtig?
    \item Welche Herausforderungen muss eine Fallauswahl bewältigen?
    \item Welche Varianten der Fallauswahl gibt es?
  \end{enumerate}
\end{frame}

\begin{frame}
\frametitle{Warum ist eine Fallauswahl wichtig?}
  \textbf{Ziel der Fallauswahl}
  \begin{itemize}
    \item Selektion von Einheiten aus einer größeren Gruppe
    \item Rückschluss von selektierten Einheiten auf Gruppe
    \item [$\rightarrow$] Fallauswahl beeinflusst Qualität der Inferenz
  \end{itemize}
  \vfill
  \textbf{Beispiel einer systematisch verzerrten Fallauswahl}
  \begin{itemize}
    \item Welche Faktoren begünstigen den Erfolg einer Revolution?
    \item Auswertung \textit{erfolgreicher} Revolutionen
    \item [$\rightarrow$] kein Rückschluss auf \textit{alle} Revolutionen möglich
  \end{itemize}
\end{frame}

\begin{frame}
\frametitle{Welche Herausford. muss eine Fallauswahl bewältigen}
  \begin{enumerate}
    \item Gewährleistung einer repräsentativen Auswahl
    \begin{itemize}
      \item Dialog Analysegesamtheit \& Population
      \item Analysegesamtheit reflektiert Theorie über Population
    \end{itemize}
    \item Varianz auf theoretisch bedeutsamen Merkmalen herstellen
    \begin{itemize}
      \item Vielfältige Testumstände erzeugen
      \item Belastbarkeit der Theorie steigt
    \end{itemize}
    \item Population eindeutig bestimmen
    \begin{itemize}
      \item Geltungsbedingungen des Populationsmodells bestimmen
      \item Problem: oft versteckte Annahmen
      \item [$\rightarrow$] Demokratien foltern nicht
    \end{itemize}
  \end{enumerate}
\end{frame}

\begin{frame}
\frametitle{Welche Varianten der Fallauswahl gibt es?}
\begin{figure}
\centering
  \begin{tikzpicture}[grow = right, scale = .5, anchor = west, edge from parent/.style={draw,-latex}, level distance = 8em]
  \tikzstyle{level 1}=[sibling distance=16em]
  \tikzstyle{level 3}=[sibling distance=2em]
  \node (root) {\itshape{Auswahl unabh. v. Anwender?}}
    child  { node (1) {Zufallsauswahl} }
    child  { node (2) {\itshape{Ziel des Anwenders?}}
      child { node (2-1) [circle, draw, fill = black, inner sep = .1em] {}
        child{ node {\dots} }
        child{ node {Diverse} }
        child{ node {Typische} }
      }
      child { node (2-2) [circle, draw, fill = black, inner sep = .1em] {}
        child{ node {\dots} }
        child{ node {Abweichende} }
        child{ node {Extreme} }
      }
    } ;
  \draw (root) edge node [sloped, anchor = south] {\scriptsize ja} (1) ;
  \draw (root) edge node [sloped, anchor = south] {\scriptsize nein} (2) ;
  \draw (2) edge node [sloped, anchor = south] {\scriptsize Konfirm.} (2-1) ;
  \draw (2) edge node [sloped, anchor = south] {\scriptsize Exploration} (2-2) ;
  \end{tikzpicture}
\end{figure}
\end{frame}

\begin{frame}
\frametitle{Allgemeine Rückmeldung zu den Diskussionspapieren}
\textbf{Zweck}
\begin{itemize}
  \item Auseinandersetzung mit einer wichtigen Kontroverse
  \item Aufarbeitung zentraler Argumente und Konfliktlinien
  \item kritische Reflektion und Systematisierung trainieren
\end{itemize}

\textbf{Häufige Stolpersteine}
\begin{itemize}
  \item \textbf{Begrifflichkeiten} Was ist eine kausale Erklärung?
  \item \textbf{Selektion} Welche Argumente in der Literatur sind relevant?
  \item \textbf{Begründungsleistung} Warum liefert ein x-zentrisches Design keine gute Einzelfallerklärung?
  \item \textbf{Positionsnahme} Nutzt euch die Unterscheidung x- und y-zentrischer Designs?
\end{itemize}
\end{frame}
\end{document}