% Preamble -------------------------------------------------
\PassOptionsToPackage{table}{xcolor}
\documentclass{beamer}
\usepackage[utf8]{inputenc}
\usepackage[ngerman]{babel}
\usepackage{adjustbox}
\usepackage{tikz}
  \usetikzlibrary{positioning, calc, decorations.pathreplacing, backgrounds}
\usepackage{multirow}
\usepackage{graphicx}
\usepackage[most]{tcolorbox}
\usepackage[style = apa, doi = false]{biblatex}
\bibliography{./library.bib}


% Slides setup ---------------------------------------------
\usetheme{Berlin}
\usecolortheme{seagull}
\usefonttheme{professionalfonts}

\title{Zusammenfassung vom 27. November 2017}
\author{Dag Tanneberg\thanks{%
  \href{mailto:dag.tanneberg@uni-potsdam.de}%
    {dag.tanneberg@uni-potsdam.de}
  }
}
\institute[Universität Potsdam]{
  {\glqq}Forschungsdesign in den Sozialwissenschaften{\grqq}\\
  Universität Potsdam\\
  Lehrstuhl für Vergleichende Politikwissenschaft\\
  Wintersemester 2017/2018
}
\date{4. Dezember 2017}

\begin{document}
\maketitle

\begin{frame}
\frametitle{Hintergrund}
\textbf{Ausgangspunkt}
\begin{itemize}
  \item Warum ist eine Forschungsfrage wichtig? \newline
  $\quad \Rightarrow$ bricht Literatur auf einen Aspekt herunter
  \item Aber wie beantworte ich eine Forschungsfrage? \newline
  $\quad \Rightarrow$ x-, y-zentrische und kontrastive Designs
\end{itemize}

\textbf{Anspruch der Unterscheidung}
\begin{itemize}
  \item legt Augenmerk auf Kausalperspektiven
  \item bietet Alternative zu Quanti/Quali
  \item ermöglicht lagerübergreifenden Diskurs
\end{itemize}
\end{frame}

\begin{frame}
\frametitle{Drei idealtypische Forschungsdesigns}

\begin{tcolorbox}[boxrule = 0pt, frame hidden, colbacktitle = gray!55, title = \textbf{x-zentriert}, height = 2cm, width = \textwidth, sharp corners, no shadow]
\begin{itemize}
  \item Wozu führt x?
  \item isoliert und schätzt einen Effekt
\end{itemize}
\end{tcolorbox}

\begin{tcolorbox}[boxrule = 0pt, frame hidden, colbacktitle = gray!55, title = \textbf{y-zentriert}, height = 2cm, width = \textwidth, sharp corners, no shadow]
\begin{itemize}
  \item Was führte zu y?
  \item sucht beste Erklärung
\end{itemize}
\end{tcolorbox}
\begin{tcolorbox}[boxrule = 0pt, frame hidden, colbacktitle = gray!55, title = \textbf{kontrastiv}, height = 2cm, width = \textwidth, sharp corners, no shadow]
\begin{itemize}
  \item Vergleich widerstreitender Theorien
  \item sucht belastbarste Theorie
\end{itemize}
\end{tcolorbox}

\end{frame}

\begin{frame}
\frametitle{Klassifikation von Forschungsdesigns}
\begin{figure}[t]
\centering
\begin{tikzpicture}[%
  level distance = 6em, sibling distance = 10em, scale = .9,%
  edge from parent/.style = {draw, -latex},%
  background rectangle/.style = {fill = gray!10},%
  show background rectangle
]
\node (0) {\itshape Anzahl der Theorien}
  child{ node (1) {x-zentriert} }
  child{
    node (2) {\itshape Verhältnis der Theorien}
    child{ node (3) {y-zentriert} }
    child{ node (4) {kontrastiv} }
  } ;
\draw node (5) [below right = 0em of 3] {} ;
\draw node (6) [left = 10em of 5] {} ;
\draw node (7) [below left = 0em of 4] {} ;
\draw node (8) [below right = 0em of 4] {} ;
\draw node (9) [left = 0em of 6] {\itshape Art der Evidenz} ;
\draw [decorate, decoration={brace, amplitude = 5pt}] (5) -- (6)
  node [midway, label = south:{\footnotesize direkt}] {}
;
\draw [decorate, decoration={brace, amplitude = 5pt}] (8) -- (7)
  node [midway, label = south:{\footnotesize direkt \& indirekt}] {}
;
\draw (0) edge node [sloped, anchor = south] {\footnotesize $1$} (1) ;
\draw (0) edge node [sloped, anchor = south] {\footnotesize $\ge 2$} (2) ;
\draw (2) edge node [sloped, anchor = south] {\scriptsize komplementär} (3) ;
\draw (2) edge node [sloped, anchor = south] {\scriptsize konkurrierend} (4) ;
\end{tikzpicture}
\end{figure}
\end{frame}

\begin{frame}
\frametitle{Stärken und Fallstricke der Designvarianten}

\begin{adjustbox}{width = \textwidth}
\begin{tabular}{*{3}{l}}
\rowcolor{gray!75}\textcolor{white} ~ & \textbf{\textcolor{white}{Stärken}} & \textbf{\textcolor{white}{Fallstricke}} \\
x-zentriert                    & transpar. Geltungsbdg.          & Relevanz isol. Detailfragen\\
~                              & direkte Evidenz                 & Einfluss konkur. Theorien?\\
\rowcolor{gray!25} y-zentriert & gute Individualerklärung        & fallabhängiges Ergebnis\\
\rowcolor{gray!25} ~           & findet beste Komb. verfügb. Th. & Th.-entw. \& -test ungetrennt\\
kontrastiv                     & verwertet indir. Evidenz        & Methoden des Theorievgl.?\\
~                              & findet beste verfügb. Th.       & Wann konkurrieren Th.?\\
\end{tabular}
\end{adjustbox}

% \begin{adjustbox}{width = \textwidth}
% \begin{tabular}{p{14.5em}p{14.5em}}
%   \multicolumn{1}{c}{\cellcolor{gray!10} \textbf{x-zentriert}} &
%   \cellcolor{blue!20}
%   \begin{tabular}{l}
%     \textbf{(x)} transpar. Geltungsbdg.\\
%     \textbf{(y)} gute Individualerklärung\\
%   \end{tabular} \\
%   \cellcolor{green!20}
%   \begin{tabular}{l}
%     \textbf{(x)} Relevanz isol. Detailfragen\\
%     \textbf{(y)} Th.-entw. \& -test ungetrennt\\
%   \end{tabular} &
%   \multicolumn{1}{c}{\cellcolor{gray!10} \textbf{y-zentriert}} \\
% \end{tabular}
% \end{adjustbox}
% \vfill

% \begin{adjustbox}{width = \textwidth}
% \begin{tabular}{p{14.5em}p{14.5em}}
%   \multicolumn{1}{c}{\cellcolor{gray!10} \textbf{x-zentriert}} &
%   \cellcolor{blue!20}
%   \begin{tabular}{l}
%     \textbf{(x)} Sichert direkte Evidenz\\
%     \textbf{(k)} Verwertet indir. Evidenz\\
%   \end{tabular} \\
%   \cellcolor{green!20}
%   \begin{tabular}{l}
%     \textbf{(x)} Einfluss konkur. Theorien?\\
%     \textbf{(k)} Methoden des Theorievgl.?\\
%   \end{tabular} &
%   \multicolumn{1}{c}{\cellcolor{gray!10} \textbf{kontrastiv}} \\
% \end{tabular}
% \end{adjustbox}
% \vfill

% \begin{adjustbox}{width = \textwidth}
% \begin{tabular}{p{14.5em}p{14.5em}}
%   \multicolumn{1}{c}{\cellcolor{gray!10} \textbf{y-zentriert}} &
%   \cellcolor{blue!20}
%   \begin{tabular}{l}
%     \textbf{(y)} beste Komb. verfügb. Th.\\
%     \textbf{(k)} findet beste verfügb. Th.\\
%   \end{tabular} \\
%   \cellcolor{green!20}
%   \begin{tabular}{l}
%     \textbf{(y)} Kein Test komplem. Th.\\
%     \textbf{(k)} Wann konkurrieren Th.?\\
%   \end{tabular} &
%   \multicolumn{1}{c}{\cellcolor{gray!10} \textbf{kontrastiv}} \\
% \end{tabular}
% \end{adjustbox}


% \begin{adjustbox}{width = \textwidth}
% \begin{tabular}{lll}
%   ~ & \multicolumn{2}{c}{\textbf{Vorteile}} \\
%   \multicolumn{1}{c}{\cellcolor{gray!10} \textbf{x-zentriert}} &
%   \cellcolor{blue!20}
%   \begin{tabular}{l}
%     \textbf{(x)} transpar. Geltungsbdg.\\
%     \textbf{(y)} gute Individualerkl.\\
%   \end{tabular} &
%   \cellcolor{blue!20}
%   \begin{tabular}{l}
%     \textbf{(x)} Sichert direkte Evidenz\\
%     \textbf{(k)} Verwertet indir. Evidenz\\
%   \end{tabular} \\
%   \hline
%   \cellcolor{green!20}
%   \begin{tabular}{l}
%     \textbf{(x)} Relevanz isol. Detailfragen\\
%     \textbf{(y)} Th.-entw. \& -test ungetrennt\\
%   \end{tabular} &
%   \multicolumn{1}{c}{\cellcolor{gray!10} \textbf{y-zentriert}} &
%   \cellcolor{blue!20}
%   \begin{tabular}{l}
%     \textbf{(y)} beste Komb. verfügb. Th.\\
%     \textbf{(k)} liefert beste verfügb. Th.\\
%   \end{tabular} \\
%   \hline
%   \cellcolor{green!20}
%   \begin{tabular}{l}
%     \textbf{(x)} Einfluss konkur. Theorien?\\
%     \textbf{(k)} Methoden des Theorievgl.?\\
%   \end{tabular} &
%   \cellcolor{green!20}
%   \begin{tabular}{l}
%     \textbf{(y)} Kein Test komplem. Th.\\
%     \textbf{(k)} Wann konkurrieren Th.?\\
%   \end{tabular} &
%   \multicolumn{1}{c}{\cellcolor{gray!10} \textbf{kontrastiv}} \\
%   \multicolumn{2}{c}{\textbf{Nachteile}} \\
% \end{tabular}
% \end{adjustbox}
\end{frame}

\begin{frame}
\frametitle{Kombinationsmöglichkeiten}
\begin{itemize}
  \item \textbf{Verallgemeinerung einer Individualerklärung}
  \begin{enumerate}
    \item Entwicklung einer y-zentrierten Erklärung
    \item nachgeordneter x-zentrierter Test einer wichtigen Implikation
    \item [$\bullet$] \scriptsize{\fullcite{Geddes.1994}}.
  \end{enumerate}
  \item \textbf{Sektoraler Vergleich von Theorien}
  \begin{enumerate}
    \item Entwicklung einer sparsamen, y-zentrierten Erklärung
    \item Komponenten nach kontrastiver Logik gerechtfertigt
    \item [$\bullet$] \scriptsize{\fullcite{Greitens.2016}}.
  \end{enumerate}
  \item \textbf{Robuste Schätzung eines Effekts}
  \begin{enumerate}
    \item Auswirkung \textit{eines} Faktors (x-zentriert)
    \item Ausschluss inkompatibler Theorien (kontrastiv)
    \item [$\bullet$] \scriptsize{\fullcite{Przeworski.1997}}.
  \end{enumerate}
\end{itemize}
\end{frame}
\end{document}