% Preamble -------------------------------------------------
\documentclass{beamer}
\usepackage[utf8]{inputenc}
\usepackage[ngerman]{babel}
\usepackage{adjustbox}
\usepackage{tikz}
  \usetikzlibrary{positioning, calc, decorations.pathreplacing, backgrounds, fit}
\usepackage{multirow}
\usepackage{graphicx}
\usepackage{caption}

\definecolor{grey538}{rgb}{240,240,240}

% Slides setup ---------------------------------------------
\usetheme{Berlin}
\usecolortheme{seagull}
\usefonttheme{professionalfonts}

\title{Zusammenfassung vom 18. Dezember 2017}
\author{Dag Tanneberg\thanks{%
  \href{mailto:dag.tanneberg@uni-potsdam.de}%
    {dag.tanneberg@uni-potsdam.de}
  }
}
\institute[Universität Potsdam]{
  {\glqq}Grundlagen der Vergleichenden Politikwissenschaft{\grqq}\\
  Universität Potsdam\\
  Lehrstuhl für Vergleichende Politikwissenschaft\\
  Wintersemester 2017/2018
}
\date{8. Januar 2018}

\begin{document}
\maketitle

\begin{frame}
  \frametitle{Leitfragen}
  \begin{enumerate}
    \item Was ist eine Verfassung?
    \item In welchen Varianten tritt sie auf?
    \item Welche Arten der Verfassungsgerichtbarkeit gibt es?
  \end{enumerate}
\end{frame}

\begin{frame}
  \frametitle{Was ist eine Verfassung?}
  \textbf{Definition}
  \begin{quote}
  A body of rules that specifies how all other legal rules
  are to be produced, applied, and interpreted.

  \footnotesize{\textnormal{(Stone Sweet, Governing with Judges, Oxford University Press, 2000)}}
  \end{quote}

  Die Verfassung
  \begin{itemize}
    \item begründet staatliche Autorität formal;
    \item definiert staatliche Institutionen;
    \item legt fest, wie staatl. Inst. miteinander \& dem Bürger umgehen.
  \end{itemize}
\end{frame}

\begin{frame}
  \frametitle{In welchen Varianten treten Verfassungen auf?}
  \begin{enumerate}
    \item Gibt es einen einzigen, fixierten Verfassungstext?
    \begin{itemize}
      \item Kodifizierte vs. unkodifizierte Verfassungen
    \end{itemize}
    \item Erfordern Verfassungsänderungen ein besonderes Verfahren?
    \begin{itemize}
      \item Verfestigte vs. flexible Verfassungen
    \end{itemize}
  \end{enumerate}

  \begin{table}
    \centering
    \caption*{\textbf{Idealtypische Verfassungen}}
    \resizebox{\textwidth}{!}{
    \begin{tabular}{*{3}{l}}
      ~& {Parlamentssuprematie} & {erhöhte Geltungskraft}\\ \hline
      Verfestigt? & nein & ja\\
      Grundrechtecharta? & nein & ja\\
      Verfassungsgerichtbarkeit? & nein & ja\\
    \end{tabular}
    }
  \end{table}
\end{frame}

\begin{frame}
\frametitle{Welche Arten der Verfassungsgerichtsbarkeit gibt es?}

\textbf{Definition}

\begin{quote}
the authority of an institution to invalidate acts of
govern\-ment, such as legislation, administrative decisions,
and judicial rulings, that violate constitutional rules

\textnormal{(Clark et al, 2014: 708)}
\end{quote}
  \begin{table}
    \centering
    \caption*{\textbf{Idealtypische Formen der Verfassungsgerichtsbarkeit}}
    \resizebox{\textwidth}{!}{
    \begin{tabular}{*{3}{l}}
      ~& {Amerikanisch} & {Europäisch}\\ \hline
      Zuständigkeit & Dezentralisiert, d.h. jedes Gericht & Zentralisiert\\
      Zeitpunkt & \textit{a posteriori} & \textit{a posteriori} \& \textit{a priori}\\
      Normenkontrolle & Konkret, d.h. laufender Rechtsstreit & Konkret \& abstrakt\\
      Antragsteller & Parteien des Rechtsstreits & Staatsorgane oder Bürger
    \end{tabular}
  }
  \end{table}
\end{frame}
\end{document}