% Preamble -------------------------------------------------
\documentclass{beamer}
\usepackage[utf8]{inputenc}
\usepackage[ngerman]{babel}
\usepackage{adjustbox}
\usepackage{tikz}
  \usetikzlibrary{positioning, calc, decorations.pathreplacing, backgrounds, fit}
\usepackage{multirow}
\usepackage{graphicx}
\usepackage{caption}

\definecolor{grey538}{rgb}{240,240,240}

% Slides setup ---------------------------------------------
\usetheme{Berlin}
\usecolortheme{seagull}
\usefonttheme{professionalfonts}

\title{Zusammenfassung vom 22. Dezember 2017}
\author{Dag Tanneberg\thanks{%
  \href{mailto:dag.tanneberg@uni-potsdam.de}%
    {dag.tanneberg@uni-potsdam.de}
  }
}
\institute[Universität Potsdam]{
  {\glqq}Grundlagen der Vergleichenden Politikwissenschaft{\grqq}\\
  Universität Potsdam\\
  Lehrstuhl für Vergleichende Politikwissenschaft\\
  Wintersemester 2017/2018
}
\date{29. Januar 2018}

\begin{document}
\maketitle

\begin{frame}
\frametitle{Frage der letzten Sitzung}
\begin{enumerate}
  \item Was will die Vetospielertheorie leisten?
  \item Wie funktioniert die Vetospielertheorie?
  \item Welche Kritik kann man an der Vetospielertheorie leisten?
\end{enumerate}
\end{frame}

\begin{frame}
\frametitle{Was will die Vetospielertheorie leisten?}
\begin{itemize}
  \item unterschiedlichste Aspekte polit. Systeme integrieren
  \begin{itemize}
    \item [z.B.] Regierungs- \& Parteiensystem, Föderalismus, etc.
  \end{itemize}
  \item sparsame Erklärung politischer Dynamiken anbieten
  \item [$\rightarrow$] betont Konstellationen von Vetospielern und Agendasetzung
\end{itemize}
\end{frame}

\begin{frame}
\frametitle{Wie funktioniert die Vetospielertheorie?}
\textbf{Allgemeine Grundlagen}
\begin{itemize}
  \item Theorie der rationalen Wahl
  \item Räumliche Modelle der Politik
  \item [$\rightarrow$] Nutzenrationale Akteure in mehrdimensionalen Situationen
  \item [$\rightarrow$] Nutzen hängt von Entfernung zu einem Idealpunkt ab
\end{itemize}
\vfill

\textbf{Besondere Grundlagen}
\begin{enumerate}
  \item Vetospieler: Akteur, der polit. Veränd. zustimmen muss
  \begin{enumerate}
    \item Konstitutionelle VS: von der Verfassung vorgesehen
    \item Parteiliche VS: entstehen im politischen Prozess
  \end{enumerate}
  \item Politikstabilität: Schwierigkeit den \textit{status quo} (SQ) zu ändern
  \begin{enumerate}
    \item Winset$_{\text{SQ}}$: Menge aller Politiken, die SQ vorgezogen werden
    \item Einstimmigkeitskern: Menge aller Politiken mit leerem Winset
  \end{enumerate}
\end{enumerate}
\end{frame}

\begin{frame}
\frametitle{Wie funktioniert die Vetospielertheorie?}
\begin{columns}
  \begin{column}{.5\textwidth}
  \begin{figure}
    \centering
    \begin{tikzpicture}[scale = 3]
% define SQ & ideal points
\coordinate (SQ) at (.2, .2) ;
\coordinate (A) at (.2, .8) ;
\coordinate (B) at (.8, .2) ;
% \coordinate (C) at (.8, .8) ;
% calculate distance from sq
\coordinate (1) at ($(SQ) - (A)$) ;
\coordinate (2) at ($(SQ) - (B)$) ;
% \coordinate (3) at ($(SQ) - (C)$) ;
% draw indifference curves
\draw let \p1 = (1) in (A) circle ({veclen(\x1,\y1)}) ;
\draw let \p1 = (2) in (B) circle ({veclen(\x1,\y1)}) ;
% \draw let \p1 = (3) in (C) circle ({veclen(\x1,\y1)}) ;
% fill winset
\begin{scope}
  \clip let \p1 = (1) in (A) circle ({veclen(\x1,\y1)}) ;
  \fill [blue!25, fill opacity = .5] let \p1 = (2) in (B) circle ({veclen(\x1,\y1)}) ;
\end{scope}
% draw unanimity core
\draw [dashed] (A) -- (B) ; % -- (C) -- (A) ;
% place visible nodes
\draw node at (A) [shape = circle, fill = black, inner sep = 0.2em, draw] {} ;
\draw node at (B) [shape = circle, fill = black, inner sep = 0.2em, draw] {} ;
% \draw node at (C) [shape = circle, fill = black, inner sep = 0.2em, draw] {} ;
\draw node at (SQ) [shape = circle, fill = black, inner sep = 0.2em, draw] {} ;
% add labels
\draw node at (A) [label = west:{A}] {} ;
\draw node at (B) [label = south:{B}] {} ;
% \draw node at (C) [label = north east:{C}] {} ;
\draw node at (SQ) [label = north east:{SQ}] {} ;
% add reference lines
\draw (0,1) [latex-] -- (0, 0) ;
\draw [-latex] (0, 0) -- (1, 0) ;
\draw node at (0, 1) [label = {north:Wirtschaft}] {} ;
\draw node at (1, 0) [label = east:Soziales] {} ;
% add legend
\draw node at (0, -.2) [fill = blue!25, shape = rectangle, label = east:Winset des Status quo] {} ;
\draw node at (0, -.4) [dashed, draw, shape = rectangle, label = east:Einstimmigkeitskern] {} ;
\end{tikzpicture}
  \end{figure}
  \end{column}
  \begin{column}{.5\textwidth}
  \textbf{Einflüsse auf Politikstabilität}
  \begin{enumerate}
    \item Anzahl: Kommen neue VS hinzu, dann sinkt die Politikstabilität nicht.
    \item [$\rightarrow$] Absorptionsregel
    \item Kongruenz: Nimmt die Entfernung zw. d. Idealpunkten zu, dann steigt die Politikstabilität.
  \end{enumerate}
  \end{column}
\end{columns}
\end{frame}

\begin{frame}
\frametitle{Letzte Bemerkungen}
\textbf{Politikstabilität beeinflusst\dots}
\begin{enumerate}
  \item Agendasetzungsmacht
  \item Regierungsstabilität
  \item Regimestabilität
  \item Unabhängigkeit der Bürokratie
  \item Unabhängigkeit der Gerichte
\end{enumerate}

\textbf{Mögliche Kritik an der Vetospielertheorie}
\begin{enumerate}
  \item VS häufig nicht leicht erkennbar
  \item Policy-Präferenzen sind exogen
  \item VS verhalten sich ausschließlich policy-seeking
\end{enumerate}
\end{frame}
\end{document}