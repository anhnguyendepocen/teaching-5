% Preamble -------------------------------------------------
\documentclass{beamer}\usepackage[]{graphicx}\usepackage[]{color}
%% maxwidth is the original width if it is less than linewidth
%% otherwise use linewidth (to make sure the graphics do not exceed the margin)
\makeatletter
\def\maxwidth{ %
  \ifdim\Gin@nat@width>\linewidth
    \linewidth
  \else
    \Gin@nat@width
  \fi
}
\makeatother

\definecolor{fgcolor}{rgb}{0.345, 0.345, 0.345}
\newcommand{\hlnum}[1]{\textcolor[rgb]{0.686,0.059,0.569}{#1}}%
\newcommand{\hlstr}[1]{\textcolor[rgb]{0.192,0.494,0.8}{#1}}%
\newcommand{\hlcom}[1]{\textcolor[rgb]{0.678,0.584,0.686}{\textit{#1}}}%
\newcommand{\hlopt}[1]{\textcolor[rgb]{0,0,0}{#1}}%
\newcommand{\hlstd}[1]{\textcolor[rgb]{0.345,0.345,0.345}{#1}}%
\newcommand{\hlkwa}[1]{\textcolor[rgb]{0.161,0.373,0.58}{\textbf{#1}}}%
\newcommand{\hlkwb}[1]{\textcolor[rgb]{0.69,0.353,0.396}{#1}}%
\newcommand{\hlkwc}[1]{\textcolor[rgb]{0.333,0.667,0.333}{#1}}%
\newcommand{\hlkwd}[1]{\textcolor[rgb]{0.737,0.353,0.396}{\textbf{#1}}}%
\let\hlipl\hlkwb

\usepackage{framed}
\makeatletter
\newenvironment{kframe}{%
 \def\at@end@of@kframe{}%
 \ifinner\ifhmode%
  \def\at@end@of@kframe{\end{minipage}}%
  \begin{minipage}{\columnwidth}%
 \fi\fi%
 \def\FrameCommand##1{\hskip\@totalleftmargin \hskip-\fboxsep
 \colorbox{shadecolor}{##1}\hskip-\fboxsep
     % There is no \\@totalrightmargin, so:
     \hskip-\linewidth \hskip-\@totalleftmargin \hskip\columnwidth}%
 \MakeFramed {\advance\hsize-\width
   \@totalleftmargin\z@ \linewidth\hsize
   \@setminipage}}%
 {\par\unskip\endMakeFramed%
 \at@end@of@kframe}
\makeatother

\definecolor{shadecolor}{rgb}{.97, .97, .97}
\definecolor{messagecolor}{rgb}{0, 0, 0}
\definecolor{warningcolor}{rgb}{1, 0, 1}
\definecolor{errorcolor}{rgb}{1, 0, 0}
\newenvironment{knitrout}{}{} % an empty environment to be redefined in TeX

\usepackage{alltt}
%\usepackage{beamerthemeshadow}
\usepackage[utf8]{inputenc}
\usepackage[ngerman]{babel}
\usepackage{hyperref}
\usepackage{graphicx}
\usepackage{tikz}
  \usetikzlibrary{positioning}
  \usetikzlibrary{calc}
  \usetikzlibrary{matrix}
\usepackage{booktabs}

% Slides setup ---------------------------------------------
\usetheme{Berlin}
\usecolortheme{seagull}
\usefonttheme{professionalfonts}
    


\title{Zusammenfassung vom 05/22/2017}
\author{Dag Tanneberg\thanks{%
  \href{mailto:dag.tanneberg@uni-potsdam.de}%
    {dag.tanneberg@uni-potsdam.de}
  }
}
\institute[Universität Potsdam]{
  {\glqq}Die politischen Dynamiken des elektoralen Autoritarismus{\grqq}\\
  Universität Potsdam\\
  Lehrstuhl für Vergleichende Politikwissenschaft\\
  Sommersemester 2017
}
\date{\today}
\IfFileExists{upquote.sty}{\usepackage{upquote}}{}
\begin{document}
\maketitle

\begin{frame}
  \frametitle{Fragestellungen}
  \begin{itemize}
    \item Wie definiert man autoritäre Herrschaft?
    \item Welche zentralen Konflikte kennzeichnen sie?
    \item Welche Anfordrg. stellt sie an die Analyse polit. Institutionen?
  \end{itemize}
\end{frame}

\begin{frame}
  \frametitle{Wie definiert man autoritäre Herrschaft?}
  \begin{block}{\textbf{Svoliks Prämissen}}
  \begin{enumerate}
    \item {\glqq}First, dictatorships inherently lack an
      independent authority with the power to enforce
      agreements among key political actors.{\grqq} (2)
    \item {\glqq}Second, violence is an ever-present and 
      ultimate arbiter of conflicts in authoritarian
      politics.{\grqq} (ibid.)
  \end{enumerate}
  \end{block}

  \textbf{Zweck von Prämissen}
  \begin{itemize}
    \item irreduzibler Ausgangspunkt der Theoriebildung
    \item Deduktion weiterer Charakteristika
    \item[$\rightarrow$] Power-sharing \& control sind Ableitungen!
  \end{itemize}
\end{frame}

\begin{frame}
  \frametitle{Welche zentralen Konflikte kennzeichnen a.H.?}
  \begin{enumerate}
    \item Problem of authoritarian control
    \begin{itemize}
      \item vertikaler Konflikt zw. Diktator \& Bürgern
      \item Machterhalt des D. gegen Btlg.-ansprüche der Bürger
      \item Kann der D. Anreize für Stillschweigen schaffen?
    \end{itemize}
    \item Problem of authoritarian power-sharing
    \begin{itemize}
      \item horizontaler Konflikt zw. Diktator \& unterstützenden Eliten
      \item Machtkonzentration vs. -dispersion
      \item Können Eliten den Opportunismus des D. sanktionieren?
    \end{itemize}
  \end{enumerate}
\end{frame}

\begin{frame}
\frametitle{Welche zentralen Konflikte kennzeichnen a.H.?}
\begin{figure}
\centering
\begin{tikzpicture}
  % Define nodes -------------------------------------------
  \node (0) {\textbf{Diktator}};
  \node (1) [right = 10em of 0] {\textbf{Eliten}} ;
  \node (2) [below = 8em of 0] {\textbf{Bevölkerung}} ;
  \node (3) at ($(0)!.5!(1)$) [label = Power-sharing] {} ;
  \node (4) at ($(0)!.5!(2)$) [label = {[rotate = 90]north:Control}] {} ;
  \node (5) [below =.25em of 3] {(polit. Institutionen)} ;
  \matrix [matrix of nodes, below of = 2, left delimiter=(,right delimiter=)] {
    Repression \\
    Kooptation \\
  } ;
  % Draw paths ---------------------------------------------
  \draw (0) [<->] to (1) ;
  \draw (0) [<->] to (2) ;
\end{tikzpicture}
\end{figure}
\end{frame}

\begin{frame}
  \frametitle{Welche Anfordrg. stellt a.H. an die Analyse \dots?}
  \begin{itemize}
    \item {\glqq}compliance with institutions is as much of
      a puzzle as are the consequences of those institutions{\grqq}
    \item[$\rightarrow$] Warum sollten sich D. an selbstgegebene Institutionen halten?
    \item[$\rightarrow$] Welche Verhaltenswirkung haben Institutionen unter a.H.?
    \item polit. Institutionen verlangen selbstdurchsetzende Äquilibria
  \end{itemize}
\end{frame}
\end{document}
