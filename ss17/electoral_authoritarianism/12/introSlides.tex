% Preamble -------------------------------------------------
\documentclass{beamer}
\usepackage[utf8]{inputenc}
\usepackage[ngerman]{babel}
\usepackage{hyperref}
\usepackage{graphicx}
\usepackage{tikz}
  \usetikzlibrary{positioning}
  \usetikzlibrary{calc}
  \usetikzlibrary{matrix}
\usepackage{adjustbox}

% Slides setup ---------------------------------------------
\usetheme{Berlin}
\usecolortheme{seagull}
\usefonttheme{professionalfonts}
    


\title{Zusammenfassung vom 07/02/2017}
\author{Dag Tanneberg\thanks{%
  \href{mailto:dag.tanneberg@uni-potsdam.de}%
    {dag.tanneberg@uni-potsdam.de}
  }
}
\institute[Universität Potsdam]{
  ``Die politischen Dynamiken des elektoralen Autoritarismus''\\
  Universität Potsdam\\
  Lehrstuhl für Vergleichende Politikwissenschaft\\
  Sommersemester 2017
}
\date{07/10/2017}

% document body --------------------------------------------
\begin{document}

\maketitle

\begin{frame}
  \frametitle{Wer geht wählen?}
  \textbf{Relevanz}: Warum an folgenlosen Wahlen teilnehmen?
  
  \textbf{Antwortvarianten}: 
  \begin{enumerate}
    \item materieller Anreiz
    \item ideologische Überzeugung
  \end{enumerate}
\end{frame}

\begin{frame}
  \frametitle{Materieller Anreiz}
  \begin{itemize}
    \item \textbf{Hintergrund}: weit verzweigert Klientelismus
    \item materielle Nutzenerwartung für Stimmabgabe
    \item ``Poor voters are more susceptible to
      clientelistic practices than wealthy voters because
      the marginal benefit of the consumption good is
      greater for them than for the wealthy.''
    \item \textbf{Hypothese}: Arme gehen häufiger zur Wahl
  \end{itemize}
\end{frame}

\begin{frame}
  \frametitle{Ideologische Überzeugung}
  \begin{itemize}
    \item \textbf{Hintergrund}: Wahlpflicht, Strafe für ungültige Stimmen
    \item Ungültige Stimme $=$ Protest oder fehlerhafte Stimmabgabe
    \item \textbf{Hypothese}: Ungültige Stimmen zahlreicher in armen und reichen Distrikten
  \end{itemize}
\end{frame}

\begin{frame}
  \frametitle{Beweisführung}
  \begin{itemize}
    \item qualitative Bestandsaufnahme
    \begin{itemize}
      \item Stimmenkauf am Wahltag: Halbierte Banknoten, Revolving ballot, Mobiltelefone
      \item Klientelismus: Wahlkreisdienste an Dorfgemeinschaften, Familiennetzwerke
      \item Zwang: Einschüchterung im Umfeld der Präsidentschaftswahl
    \end{itemize}
    \item Quantitative Bestandsaufnahme
    \begin{itemize}
      \item Auswertung von Wahlkreisergebnissen
      \item Schätzung des Anteils wählender Armer aus abgegebenen Stimmen und Illiteraten im Wahlkreis
    \end{itemize} 
  \end{itemize}
\end{frame}

\begin{frame}
  \frametitle{Kritik}
  \begin{itemize}
    \item theoretische Setzung: Warum machen Arme mehr Fehler bei der Stimmabgabe?
    \item Operationalisierung: Analphabetismus $\neq$ Armut
    \item Methode: Gefahr des ökologischen Fehlschlusses
  \end{itemize}
\end{frame}

\end{document}