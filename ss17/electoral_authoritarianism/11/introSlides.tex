% Preamble -------------------------------------------------
\documentclass{beamer}\usepackage[]{graphicx}\usepackage[]{color}
%% maxwidth is the original width if it is less than linewidth
%% otherwise use linewidth (to make sure the graphics do not exceed the margin)
\makeatletter
\def\maxwidth{ %
  \ifdim\Gin@nat@width>\linewidth
    \linewidth
  \else
    \Gin@nat@width
  \fi
}
\makeatother

\definecolor{fgcolor}{rgb}{0.345, 0.345, 0.345}
\newcommand{\hlnum}[1]{\textcolor[rgb]{0.686,0.059,0.569}{#1}}%
\newcommand{\hlstr}[1]{\textcolor[rgb]{0.192,0.494,0.8}{#1}}%
\newcommand{\hlcom}[1]{\textcolor[rgb]{0.678,0.584,0.686}{\textit{#1}}}%
\newcommand{\hlopt}[1]{\textcolor[rgb]{0,0,0}{#1}}%
\newcommand{\hlstd}[1]{\textcolor[rgb]{0.345,0.345,0.345}{#1}}%
\newcommand{\hlkwa}[1]{\textcolor[rgb]{0.161,0.373,0.58}{\textbf{#1}}}%
\newcommand{\hlkwb}[1]{\textcolor[rgb]{0.69,0.353,0.396}{#1}}%
\newcommand{\hlkwc}[1]{\textcolor[rgb]{0.333,0.667,0.333}{#1}}%
\newcommand{\hlkwd}[1]{\textcolor[rgb]{0.737,0.353,0.396}{\textbf{#1}}}%
\let\hlipl\hlkwb

\usepackage{framed}
\makeatletter
\newenvironment{kframe}{%
 \def\at@end@of@kframe{}%
 \ifinner\ifhmode%
  \def\at@end@of@kframe{\end{minipage}}%
  \begin{minipage}{\columnwidth}%
 \fi\fi%
 \def\FrameCommand##1{\hskip\@totalleftmargin \hskip-\fboxsep
 \colorbox{shadecolor}{##1}\hskip-\fboxsep
     % There is no \\@totalrightmargin, so:
     \hskip-\linewidth \hskip-\@totalleftmargin \hskip\columnwidth}%
 \MakeFramed {\advance\hsize-\width
   \@totalleftmargin\z@ \linewidth\hsize
   \@setminipage}}%
 {\par\unskip\endMakeFramed%
 \at@end@of@kframe}
\makeatother

\definecolor{shadecolor}{rgb}{.97, .97, .97}
\definecolor{messagecolor}{rgb}{0, 0, 0}
\definecolor{warningcolor}{rgb}{1, 0, 1}
\definecolor{errorcolor}{rgb}{1, 0, 0}
\newenvironment{knitrout}{}{} % an empty environment to be redefined in TeX

\usepackage{alltt}
\usepackage[utf8]{inputenc}
\usepackage[ngerman]{babel}
\usepackage{hyperref}
\usepackage{graphicx}
\usepackage{tikz}
  \usetikzlibrary{positioning}
  \usetikzlibrary{calc}
  \usetikzlibrary{matrix}
\usepackage{adjustbox}

% Slides setup ---------------------------------------------
\usetheme{Berlin}
\usecolortheme{seagull}
\usefonttheme{professionalfonts}
    


\title{Zusammenfassung vom 06/26/2017}
\author{Dag Tanneberg\thanks{%
  \href{mailto:dag.tanneberg@uni-potsdam.de}%
    {dag.tanneberg@uni-potsdam.de}
  }
}
\institute[Universität Potsdam]{
  ``Die politischen Dynamiken des elektoralen Autoritarismus''\\
  Universität Potsdam\\
  Lehrstuhl für Vergleichende Politikwissenschaft\\
  Sommersemester 2017
}
\date{07/03/2017}

% document body --------------------------------------------
\IfFileExists{upquote.sty}{\usepackage{upquote}}{}
\begin{document}

\frame{\titlepage}

\begin{frame}
\frametitle{Wie wird gewählt?}

Elektoraler Autoritarismus unterminiert systematisch und in
verschiedensten Varianten den Wettbewerbsgrad von Wahlen.

\begin{enumerate}
  \item Welche Rolle spielt Gewalt?
  \begin{itemize}
    \item ständige Möglichkeit unter autoritärer Herrschaft
    \item Repression $=$ Gewalt als Herrschaftsmittel
  \end{itemize}
  \item Welche Rolle spielen politische Institutionen?
  \begin{itemize}
    \item nominell demokr. Inst. $=$ Instrumente autorit. Herrschaft
    \item Wie erreicht man diese Transmutation?
  \end{itemize}
\end{enumerate}
\end{frame}

\begin{frame}
  \frametitle{Norma Kriger, ZANU(PF) Strategies}
  Anhänger und Kandidaten der Opposition werden vor und nach
  der Wahl Opfer systematischer Diskriminierung.
  \vfill
  \textbf{Politischer Diskurs}
  \begin{itemize}
    \item Androhung \& Rechtfertigung von Gewalt
    \item Diffamierung polit. Opposition als Staatsfeinde
  \end{itemize}
  \vfill
  \textbf{Anwendung von Gewalt}
  \begin{itemize}
    \item Anwerbung jugendl. Schlägertrupps (Ausnutzung sozialer Not)
    \item körperliche Misshandlung \& Entzug von Lebenschancen
    \item Zwangsmitgliedschaften in der ZANU(PF)
  \end{itemize}
\end{frame}

\begin{frame}
  \frametitle{Golosov, Authoritarian Electoral Engineering}
  \begin{itemize}
    \item \textbf{Hintergrund} Rezentralisierung d. russ. Staates unter Putin
    \item \textbf{Zielkonflikt} Autoritäre Macht vs. Demokrat. Legitimität
    \item \textbf{polit. Instrument} Verrechnung von Stimmen in Sitze
    \begin{itemize}
      \item vor 2007 \href{https://de.wikipedia.org/wiki/Hare-Niemeyer-Verfahren}{\textit{Hare-Niemayer-Verfahren}}
      \item nach 2007: Imperiali Highest Averages (IHA) empfohlen
      \item i.d.R. Tyumen (IHA, aber 1 Sitz für jede Partei $\geq 7\%$)
    \end{itemize}
  \end{itemize}
  \begin{quote}
  \textnormal{%
    \scriptsize{%
      ``Thus in fact, the political conditions of electoral
      authoritarianism generate two mutually contradictory
      sets of incentives for electoral engineering. On the
      one hand, there are incentives for maintaining the
      political monopoly, which required providing the
      pro-government party with super-majorities previously.
      On the other hand, there are incentives for securing
      the democratic appearance of the regime and co-opting
      the opposition, which makes it imperative to ensure
      that licensed opposition parties are represented. (1623)
    }
  }
  \end{quote}
\end{frame}

\end{document}
