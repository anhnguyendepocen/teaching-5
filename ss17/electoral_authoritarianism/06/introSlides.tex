% Preamble -------------------------------------------------
\documentclass{beamer}\usepackage[]{graphicx}\usepackage[]{color}
%% maxwidth is the original width if it is less than linewidth
%% otherwise use linewidth (to make sure the graphics do not exceed the margin)
\makeatletter
\def\maxwidth{ %
  \ifdim\Gin@nat@width>\linewidth
    \linewidth
  \else
    \Gin@nat@width
  \fi
}
\makeatother

\definecolor{fgcolor}{rgb}{0.345, 0.345, 0.345}
\newcommand{\hlnum}[1]{\textcolor[rgb]{0.686,0.059,0.569}{#1}}%
\newcommand{\hlstr}[1]{\textcolor[rgb]{0.192,0.494,0.8}{#1}}%
\newcommand{\hlcom}[1]{\textcolor[rgb]{0.678,0.584,0.686}{\textit{#1}}}%
\newcommand{\hlopt}[1]{\textcolor[rgb]{0,0,0}{#1}}%
\newcommand{\hlstd}[1]{\textcolor[rgb]{0.345,0.345,0.345}{#1}}%
\newcommand{\hlkwa}[1]{\textcolor[rgb]{0.161,0.373,0.58}{\textbf{#1}}}%
\newcommand{\hlkwb}[1]{\textcolor[rgb]{0.69,0.353,0.396}{#1}}%
\newcommand{\hlkwc}[1]{\textcolor[rgb]{0.333,0.667,0.333}{#1}}%
\newcommand{\hlkwd}[1]{\textcolor[rgb]{0.737,0.353,0.396}{\textbf{#1}}}%
\let\hlipl\hlkwb

\usepackage{framed}
\makeatletter
\newenvironment{kframe}{%
 \def\at@end@of@kframe{}%
 \ifinner\ifhmode%
  \def\at@end@of@kframe{\end{minipage}}%
  \begin{minipage}{\columnwidth}%
 \fi\fi%
 \def\FrameCommand##1{\hskip\@totalleftmargin \hskip-\fboxsep
 \colorbox{shadecolor}{##1}\hskip-\fboxsep
     % There is no \\@totalrightmargin, so:
     \hskip-\linewidth \hskip-\@totalleftmargin \hskip\columnwidth}%
 \MakeFramed {\advance\hsize-\width
   \@totalleftmargin\z@ \linewidth\hsize
   \@setminipage}}%
 {\par\unskip\endMakeFramed%
 \at@end@of@kframe}
\makeatother

\definecolor{shadecolor}{rgb}{.97, .97, .97}
\definecolor{messagecolor}{rgb}{0, 0, 0}
\definecolor{warningcolor}{rgb}{1, 0, 1}
\definecolor{errorcolor}{rgb}{1, 0, 0}
\newenvironment{knitrout}{}{} % an empty environment to be redefined in TeX

\usepackage{alltt}
\usepackage[utf8]{inputenc}
\usepackage[ngerman]{babel}
\usepackage{hyperref}
\usepackage{graphicx}
\usepackage{tikz}
  \usetikzlibrary{positioning}
  \usetikzlibrary{calc}
  \usetikzlibrary{matrix}
\usepackage{adjustbox}

% Slides setup ---------------------------------------------
\usetheme{Berlin}
\usecolortheme{seagull}
\usefonttheme{professionalfonts}
    


\title{Zusammenfassung vom 05/15/2017}
\author{Dag Tanneberg\thanks{%
  \href{mailto:dag.tanneberg@uni-potsdam.de}%
    {dag.tanneberg@uni-potsdam.de}
  }
}
\institute[Universität Potsdam]{
  ``Die politischen Dynamiken des elektoralen Autoritarismus''\\
  Universität Potsdam\\
  Lehrstuhl für Vergleichende Politikwissenschaft\\
  Sommersemester 2017
}
\date{\today}

% document body --------------------------------------------
\IfFileExists{upquote.sty}{\usepackage{upquote}}{}
\begin{document}

\frame{\titlepage}

\begin{frame}
  \frametitle{Fragestellungen der vergangenen Sitzung}
  \begin{enumerate}
    \item Helfen uns demokratische Wahlen, Autokratien zu erkennen?
    \item Warum sind Wahlen im elekt. Autoritarismus wichtig?
  \end{enumerate}
\end{frame}

\begin{frame}
  \frametitle{Helfen uns demokratische Wahlen, \dots?}
  \textbf{Klassifikation politischer Regime nach Przeworski et al.}
  \begin{itemize}
    \item minimaler, prozeduraler Demokratiebegriff
    \item dichotome Unterscheidung nach streng objektiven Merkmalen
    \item explizite Darstellung systematischer Messfehler
  \end{itemize}

  \begin{quote}
    \normalfont
    \small
    ``Our purpose is to distinuish between (1) regimes that
    allow some, even if limited, regularized competition
    among conflicting visions and interests and (2) regimes
    in which some values or interests enjoy a monopoly
    buttressed by the threat or the actual use of force.
    Thus `democracy', for us, is a regime in which those who
    govern are selected through contested elections. This
    definition has two parts: `government' and
    `contestation'.'' (S. 15)
  \end{quote}
\end{frame}

\begin{frame}
  \frametitle{Helfen uns demokratische Wahlen, \dots?}
  \begin{adjustbox}{max totalsize={\textwidth}{.9\textheight}, center}
\begin{tikzpicture}[
  every matrix/.append style={ampersand replacement=\&,matrix of nodes},
  edge from parent/.append style={-latex}
]
% --- Layout & Styles --------------------------------------
\tikzstyle{solid node}=[circle,draw,inner sep=1.2,fill=black];
\tikzstyle{hollow node}=[circle,draw,inner sep=1.2];
\tikzstyle{level 1}=[sibling distance = 18em]
\tikzstyle{level 2}=[sibling distance = 7em]
% --- Tree -------------------------------------------------
\node(0){Democracy}
  child{
    node(0-1){Government}
      child{node(0-1-1){Executive}
        child{node{\sc \#1 exselec}}
      }
      child{node(0-1-2){Legislative}
        child{node{\sc \#2 legselec}}
      }
  }
  child{
    node(0-2){Contestation}
    child{ node(0-2-1) [matrix] {\textit{Ex ante} \\ uncertainty \\}
      child{node{\sc \#3 party}}
    }
    child{ node(0-2-2) [matrix] {\textit{Ex post} \\ irreversibility \\}
      child{node{\sc \#4 typeii}}
    }
    child{ node(0-2-3) [matrix] {Repeata- \\ bility \\}
      child{node{\sc \#3 incumb}}
    }
  } ;
\end{tikzpicture}
\end{adjustbox}
\end{frame}

\begin{frame}
  \frametitle{Helfen uns demokratische Wahlen, \dots?}
  \textbf{ {\sc typeii} Ärger um den Regierungswechsel}
  \begin{itemize}
    \item Lässt eine Regierung nur wählen, weil sie nicht verliert?
    \item[$\rightarrow$] Objektive Merkmale (\#1--3) reichen nicht aus
    \item ``Err we must; the question is which way.'' (23)
    \begin{itemize}
      \item Typ 1 Wenn im Zweifel, dann Autokratie.
      \item Typ 2 Wenn im Zweifel, dann Demokratie.
    \end{itemize}
    \item ``We choose to take a cautious stance, that is, to avoid type-II errors.'' (25)
  \end{itemize}
\end{frame}

\begin{frame}
  \frametitle{Warum sind Wahlen im elekt. Autoritarismus wichtig?}
  \textbf{Schedlers Kritik an Przeworski et al.}
  \begin{itemize}
    \item Subj. Urteile unverzichtbar zur Bewertung obj. Merkmale
    \item[$\rightarrow$] Wahlmanipulation selten objektiver Fakt
    \begin{enumerate}
      \item Wahlmanipulation i.d.R. vielfältig \& dezentralisiert
      \item autorit. Herrschaft überformt Präferenzgenese \& -äußerung
    \end{enumerate}
    \item Erhebung des autorit. Gehalts einer Wahl durch subj. Urteilen
  \end{itemize}
\end{frame}

\begin{frame}
  \frametitle{Warum sind Wahlen im elekt. Autoritarismus wichtig?}
  \textbf{Was ist dann elektoraler Autoritarismus?}
  \begin{itemize}
    \item neuer Typ Autoritarismus
    \item betont Zugang zu polit. Macht \& Einfluss
    \item beschreibt die Öffnung autoritärer Herrschaft für
    \begin{enumerate}
      \item (semi-)kompetitive Wahlen zu exek. \& legisl. Ämtern
      \item politischem \& sozialem Pluralismus
      \item organisierte Dissidenz
    \end{enumerate}
  \end{itemize}
\end{frame}

\begin{frame}
  \frametitle{Warum sind Wahlen im elekt. Autoritarismus wichtig?}
  \textbf{Wahlen im EA sind konstitutiv für}
  \begin{enumerate}
    \item \textbf{Nested games} (1) Manipulation \& (2) institutionelle Reform
    \item \textbf{Bürger} Primat demokrat. Legitimation i.S.v. Zustimmung
    \item \textbf{teilautonome Opposition} (1) Partizipation oder Boykott, (2) Akzeptanz oder Protest
    \item \textbf{herrschende Parteien} Mobilisierung, Patronage \& Konfliktschlichtung
  \end{enumerate}
\end{frame}


\end{document}
