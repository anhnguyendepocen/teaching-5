% Preamble -------------------------------------------------
\documentclass{beamer}
\usepackage[german]{babel}
\usepackage{bibentry}

\addbibresource{./library.bib}

\definecolor{grey538}{RGB}{240,240,240}

\newcommand*{\priority}[1]{\begin{tikzpicture}[scale=0.15]%
    \draw (0,0) circle (1);
    \fill[fill opacity=.4,fill=blue] (0,0) -- (90:1) arc (90:90-#1*3.6:1) -- cycle;
    \end{tikzpicture}}
% document body --------------------------------------------
\begin{document}

\maketitle

\begin{frame}
    \frametitle{Leitfragen der Sitzung}
    \begin{enumerate}
        \item Was zeichnet eine politische Partei aus?
        \item Was kennzeichnet ein Parteiensystem?
        \item Wie entstehen oder verändern sich Parteiensysteme?
    \end{enumerate}
\end{frame}

\begin{frame}
    \frametitle{Was zeichnet eine politische Partei aus?}
    \begin{itemize}
        \item \textbf{Definitionsversuch}\newline
        ``eine \textit{Gruppe gleichgesinnter Personen}, die sich in
        unterschiedlicher organisatorischer Form an der politischen
        Willensbildung beteiligt und danach \textit{strebt,
        politische Positionen zu besetzen und ihre Ziele in einem
        Gemeinwesen durchzusetzen}.'' (Winkler 2010: 216; mein
        Ausdruck)
    \item \textbf{Parteifunktionen}\newline
        Kommunikation \& Responsivität; Rekrutierung
        politischen Personals; Aggregation und Artikulation
        politischer Interessen (Programmbildung); Regierungsbildung
        und -stützung u.\,a.\,m.
    \end{itemize}
\end{frame}

\begin{frame}
    \frametitle{Was kennzeichnet ein Parteiensystem}
    \begin{itemize}
        \item \textbf{Definitionsversuch}\newline
        ``Menge von Parteien und die zwischen ihnen und ihren Eigenschaften bestehenden relevanten Beziehungen'' (Winkler 2010: 226)
        \begin{itemize}
            \item [$\rightarrow$] i.\,d.\,R. Wettbewerb (CDU/CSU vs. SPD)
            \item [$\rightarrow$] auch Kooperation möglich (CDU und CSU)
        \end{itemize}
        \item \textbf{Eigenschaften eines Parteiensystems}
        \begin{itemize}
            \item üblich: Format, d.\,h. effekt. Parteienzahl ($\sum_{i=1}^N p_i^{-2}$)
            \item weitere Möglichkeiten: Volatilität, Polarisierung
        \end{itemize}
    \end{itemize}
\end{frame}

\begin{frame}
    \frametitle{Wie entstehen oder verändern sich Parteiensysteme?}
    \begin{itemize}
        \item Social Cleavages nach Stein/Rokkan
        \begin{itemize}
            \item tiefgreifende gesellschaftl. Konfliktlinie
            \item reicht zur Organisation polit. Identitäten
            \item Klassische Konfliktlinien: Zentrum vs. Peripherie;
                Staat vs. Kirche; Kapital vs. Arbeit; Landwirtschaft
                vs. Industrie
        \end{itemize}
        \item Was sagt das über Parteiensysteme aus?
        \begin{itemize}
            \item Parteien beuten existierende Konfliktstrukturen aus\newline
            $\rightarrow$ primordiale Konflikte
            \item Alternativ: politische Unternehmer aktivieren Konflikte\newline
            $\rightarrow$ instrumentelle Konflikte
            \item bottom-up/Nachfrageseite vs. top-down/Angebotsseite
        \end{itemize}


    \end{itemize}
\end{frame}
\end{document}
