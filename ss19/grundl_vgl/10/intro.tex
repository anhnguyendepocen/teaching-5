% Preamble -------------------------------------------------
\documentclass{beamer}
\usepackage[german]{babel}
\usepackage{bibentry}

\addbibresource{./library.bib}

\definecolor{grey538}{RGB}{240,240,240}

\newcommand*{\priority}[1]{\begin{tikzpicture}[scale=0.15]%
    \draw (0,0) circle (1);
    \fill[fill opacity=.4,fill=blue] (0,0) -- (90:1) arc (90:90-#1*3.6:1) -- cycle;
    \end{tikzpicture}}
% slides ---------------------------------------------------
\begin{document}
\maketitle

\begin{frame}
\frametitle{Ausgangspunkt}

\textbf{Wahlen in Demokratien}
\begin{itemize}
  \item stellen allgemeinste Partizipationsform dar
  \item legitimieren Herrschaft in repr. Demokratien
\end{itemize}

\textbf{Leitfragen der Sitzung}
\begin{enumerate}
  \item Wahlsysteme: Definitionen, Bausteine, Varianten
  \item Welche Effekte hat ein Wahlsystem?
  \item Methodologische Probleme der Wahlsystemforschung
\end{enumerate}
\end{frame}

\begin{frame}
\frametitle{Politische Bdtg. von Wahlsystemen}
\textbf{Das Wahlsystem}
\begin{itemize}
  \item \textbf{Def.}: Regeln zur Übertragung von Päferenzen in Stimmen und Sitze
  \item \textbf{Zweck}: Rekrutierung polit. Ämter \& Repräsentativvers.
\end{itemize}
\textbf{Zielkonflikt}
\begin{itemize}
  \item Proportionales Ergebnis vs. Verantwortung f. Entscheidungen
  \item Wie viele polit. Parteien sollen in das Parlament einziehen werden?
  \begin{enumerate}
    \item So viele wie nötig. $\rightarrow$ Geringe Disproportionalität
    \item So wenig wie möglich. $\rightarrow$ Eindeutige polit. Verantwortung
  \end{enumerate}
\end{itemize}
\end{frame}

\begin{frame}
\frametitle{Elementare Bausteine eines Wahlsystems}

Der Charakter eines Wahlsystems resultiert aus dem Zusammenspiel verschiedener Bausteine.

\begin{enumerate}
  \item \textbf{Wahlkreis}: Einheit, in der Stimmen in Mandate übertragen werden
  \begin{enumerate}
    \item Anzahl zu vergebender Mandate
    \item Anzahl stimmberechtigter Bürger
  \end{enumerate}
  \item \textbf{Form der Kandidatur}: Individual- vs. Listenkandidatur
  \item \textbf{Stimmgebungsverfahren}:
  \begin{enumerate}
    \item Anzahl zu vergebender Stimmen: Einzel- vs. Mehrstimmgebung
    \item Art der Stimmgebung: Kandidaten- vs. Listenstimmgebung
  \end{enumerate}
  \item \textbf{Stimmverrechnungsverfahren}
  \begin{enumerate}
    \item Mehrheitswahl: relativ oder absolut
    \item Verhältniswahl: Wahlzahlverfahren, Divisorverfahren, STV
  \end{enumerate}
\end{enumerate}
\end{frame}

\begin{frame}
  \frametitle{Wie arbeitet ein Wahlsystem?}
  \begin{enumerate}
    \item \textbf{mechanischer Effekt}: technische Regeln d. Mandatsvergabe
    \begin{itemize}
      \item Wie viele Parteien erringen ein Mandat?
      \item Wahlkreisgröße zentral
    \end{itemize}
    \item \textbf{psycholog. Effekt}: Antizipation von 1 durch Wähler \& Eliten
    \begin{itemize}
      \item Wie viele Parteien bewerben sich auf ein Mandat?
      \item Wählen die Bürger strategisch?
    \end{itemize}
  \item [$\rightarrow$] psycholog. Effekt setzt den mechanischen E. voraus
  \item [$\rightarrow$] Effekte können in der Realität kaum getrennt werden
  \end{enumerate}
\end{frame}

\end{document}