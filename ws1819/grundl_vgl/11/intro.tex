% Preamble -------------------------------------------------
\PassOptionsToPackage{table}{xcolor}
\documentclass{beamer}
\usepackage[utf8]{inputenc}
\usepackage[ngerman]{babel}
\usepackage{adjustbox}
\usepackage{tikz}
  \usetikzlibrary{positioning, calc, decorations.pathreplacing, backgrounds, fit}
\usepackage{multirow}
\usepackage{graphicx}
\usepackage[most]{tcolorbox}
\usepackage{booktabs}

% Slides setup ---------------------------------------------
\usetheme{Berlin}
\usecolortheme{seagull}
\usefonttheme{professionalfonts}

\title{Zusammenfassung vom 14. Januar 2019}
\author{Dag Tanneberg\thanks{%
  \href{mailto:dag.tanneberg@uni-potsdam.de}%
    {dag.tanneberg@uni-potsdam.de}
  }
}
\institute[Universität Potsdam]{
  {\glqq}Grundlagen der Vergleichenden Politikwissenschaft{\grqq}\\
  Universität Potsdam\\
  Lehrstuhl für Vergleichende Politikwissenschaft\\
  Wintersemester 2018/2019
}
\date{\today}

% slides ---------------------------------------------------
\begin{document}
\maketitle

\begin{frame}
\frametitle{Leitfragen der Sitzung}
\begin{enumerate}
  \item Warum sollten wir über Regierungssysteme sprechen?
  \item Welche wesentlichen Regierungssysteme gibt es?
  \item Was bedeuten deren Unterschiede für Parteien?
\end{enumerate}
\end{frame}

\begin{frame}
\frametitle{Warum sollten wir über Regierungssysteme sprechen?}
\begin{itemize}
  \item \textbf{Repräsentative Demokratie}: Delegation polit. Macht
  \item \textbf{Problem}: Wie zieht man eine Regierung zur Verantwortung?
  \item \textbf{Varianten repräsentativer Demokratie}:
\end{itemize}
\begin{columns}
\begin{column}{.5\textwidth}
\begin{block}{\textbf{1. Parlamentarismus}}
\begin{tikzpicture}[grow = up, edge from parent/.style={draw,-latex}]
\node (0) [shape = rectangle, fill = white] {Staatsbürger}
  child{ node (1) [shape = rectangle, fill = white] {Parlament}
    child{node (2) [shape = rectangle, fill = white] {Regierung} }
  } ;
\node (3) at ($(1)!0.5!(2)-(2.5,0)$) [shape = rectangle, fill = white] {Verwaltung};
\node (l1) at ($(0)!0.5!(1)+(.6,0)$) {\small wählt};
\node (l2) at ($(1)!0.5!(2)+(1.1,0)$) {
  \small
  \begin{tabular}{l}
    hält polit.\\
    verantwortl.
  \end{tabular}
};
\node (l1) at ($(2)!0.5!(3)-(0.5,-.25)$) {\small leitet};
\draw [-latex] (2)--(3);
\end{tikzpicture}
\end{block}
\end{column}
\begin{column}{.5\textwidth}
\begin{block}{\textbf{2. Präsidentialismus}}
\begin{tikzpicture}[grow = up, sibling distance = 6em, edge from parent/.style={draw,-latex}]
\node (0) [shape = rectangle, fill = white] {Staatsbürger}
  child [level distance = 8em] { node (1) [shape = rectangle, fill = white] {Parlament} }
  child [level distance = 8em] { node (2) [shape = rectangle, fill = white] {Präsident}
    child [grow = south west, level distance = 3.5em] { node (3) [shape = rectangle, fill = white] {Kabinett}
      child [grow = south] {node (4) [shape = rectangle, fill = white] {Verwaltung} }
    }
  } ;
\node (l1) at ($(0)!0.5!(1)-(.6,0)$) {\small wählt} ;
\node (l2) at ($(2)!0.5!(3)-(0.7,-0.05)$) {\small ernennt} ;
\node (l3) at ($(3)!0.5!(4)-(0.5,0)$) {\small leitet} ;
\end{tikzpicture}
\end{block}
\end{column}
\end{columns}
\end{frame}

\begin{frame}
\frametitle{Welche wesentlichen Regierungssysteme gibt es?}
\begin{table}
  \centering
  \caption{Varianten geschlossener Exekutiven}
  \begin{tabular}{*{4}{l}}
    \toprule
     ~ & \multicolumn{2}{c}{Polit. Abberufbarkeit der Regierung} \\
     ~ & \multicolumn{2}{c}{durch das Parlament} \\
    Auswahl d. Reg. & \multicolumn{1}{c}{Ja} & \multicolumn{1}{c}{Nein} \\
    \cmidrule{2-3}
    Parlament & \textit{parlamentarisch} & versammlungsunabh.\\
    Volk & direktwahl-parlament. & \textit{präsidentiell}\\
    \bottomrule
  \end{tabular}
\end{table}
\end{frame}

\begin{frame}
\frametitle{Was bedeuten deren Unterschiede für Parteien?}
\begin{itemize}
  \item Parteien lösen Probleme kollektiven Handelns
  \begin{enumerate}
    \item stabilisieren parlamentarische Entscheidungen
    \item leihen Kandidaten ihre Reputation
  \end{enumerate}
  \item [$\rightarrow$] bedürfen kontinuierlicher Pflege durch die Parteiführung
  \item Direktwahl der Exek.: Präsidentialisierung politischer Parteien
  \item [$\rightarrow$] Parteien passen ihre Struktur der Präsidentschaftswahl an
  \item [$\rightarrow$] Gewinn der Präsidentschaft $=$ Pfründe $+$ Politikgestaltung
  \item Parteien im Präsidentialismus\dots
  \begin{enumerate}
    \item sind tendenziell programmatisch weniger geschlossen.
    \item sind tendenziell lockerer organisiert.
    \item suchen eher den Konflikt mit dem Präsidenten.
  \end{enumerate}
\end{itemize}
\end{frame}
\end{document}