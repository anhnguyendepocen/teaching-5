% Preamble -------------------------------------------------
\documentclass{beamer}
\usepackage[utf8]{inputenc}
\usepackage[ngerman]{babel}
\usepackage{adjustbox}
\usepackage{tikz}
  \usetikzlibrary{positioning, calc, decorations.pathreplacing, backgrounds, fit}
\usepackage{multirow}
\usepackage{graphicx}
\usepackage{caption}

\definecolor{grey538}{rgb}{240,240,240}

% Slides setup ---------------------------------------------
\usetheme{Berlin}
\usecolortheme{seagull}
\usefonttheme{professionalfonts}

\title{Zusammenfassung vom 19. November 2018}
\author{Dag Tanneberg\thanks{%
  \href{mailto:dag.tanneberg@uni-potsdam.de}%
    {dag.tanneberg@uni-potsdam.de}
  }
}
\institute[Universität Potsdam]{
  {\glqq}Grundlagen der Vergleichenden Politikwissenschaft{\grqq}\\
  Universität Potsdam\\
  Lehrstuhl für Vergleichende Politikwissenschaft\\
  Wintersemester 2018/19
}
\date{26. November 2018}

\begin{document}
\maketitle

\begin{frame}
  \frametitle{Leitfragen}
  \begin{enumerate}
    \item Was kennzeichnet eine (parlamentarische) Regierung?
    \item Wie kann ich eine Regierungszusammensetzung vorhersagen?
    \item Woran scheitern diese Vorhersagen?
  \end{enumerate}
\end{frame}

\begin{frame}
  \frametitle{Was kennzeichnet (parlamentarische) Regierungen?}
  \textbf{Aufbau und Arbeitsweise}
  \begin{itemize}
    \item Parlamentarische Regierung: Premier Minister + Kabinett, d.i. Minister
    \item Ressortprinzip: Minister führen ihr Haus eigenverantwortlich im Rahmen der gemeinsam festgelegten Richtlinien
    \item Kollektivorgan, d.\,h. Kabinettsmitglieder verantworten gemeinsam Regierungsbeschlüsse und kritisieren nicht öffentlich die Regierungsarbeit
  \end{itemize}
  \textbf{Randbedingungen}
  \begin{itemize}
      \item Misstrauensvotum
      \item (oft auch) Investiturabstimmung
  \end{itemize}
\end{frame}

\begin{frame}
  \frametitle{Wie kann ich eine Regierungszus. vorhersagen?}
  \begin{enumerate}
    \item \textbf{office seeking}
    \begin{itemize}
      \item Politiker streben nur nach Ämtern
      \item Koalitionen: Kooperation durch Abgabe von Ämtern erkaufen
      \item \textbf{Hypothesen}
      \begin{enumerate}
        \item führt zu Minimal Winning Coalitions
        \item $\%$ Kabinettsposten $\sim$ $\%$ Mandaten an der Regierung
        \item [$\rightarrow$] Gamson's Law
      \end{enumerate}
    \end{itemize}
    \item \textbf{policy seeking}
    \begin{itemize}
      \item Politiker streben nur nach Politikinhalten
      \item Koalitionen: Kooperation durch Inhalte erkauft
      \item \textbf{Hypothese} führt zu Minimal Connected Winning Coalitions
    \end{itemize}
  \end{enumerate}
\end{frame}

\begin{frame}
  \frametitle{Woran scheitern diese Vorhersagen?}
  \begin{enumerate}
    \item \textbf{Minderheitsregierungen}
    \begin{itemize}
      \item institutioneller \& politischer Rahmen unberücksichtigt
      \item Institutionen: starke Ausschüsse, keine Investiturabstimmung
      \item Kontext: starke Partei, Korporatismus
    \end{itemize}
    \item \textbf{Übergroße Mehrheiten}
    \begin{itemize}
      \item Ausnahmesituationen \& komplementäre politische Erwägungen unberücksichtigt
      \item Ausnahmesituationen: z.\,B. Regierungen der nationalen Einheit
      \item kompl. polit. Erw.: Disziplinierung kleiner Koalitionspartner
    \end{itemize}
  \end{enumerate}
\end{frame}

\end{document}
